\section{Quantitative Model}
\label{sec:quantitative_model}

\subsection{Environment}

\subsubsection{Time and Agents}
Time is discrete. A continuum of firms $j \in [0,1]$ faces idiosyncratic productivity $z_{j,t}$. Each firm holds tangible capital $K_{j,t}$, intangible capital $S_{j,t}$, and debt $D_{j,t}$. Firms exit with probability $\zeta \in (0,1)$; entrants draw $z_0 \sim F_z$ on $[\underline{z}, \overline{z}]$ and start with $K_0 = \bar{K}$, $S_0 = \bar{S}$, $D_0 = 0$, where $\bar{K} = \E_{\Psi^*}[K]$ and $\bar{S} = \E_{\Psi^*}[S]$ are the cross-sectional means from the stationary distribution.

\subsubsection{Household}
A representative household owns all firms, supplies skilled and unskilled labor inelastically, and has standard preferences over consumption:
\[
\E_0 \sum_{t=0}^{\infty} \beta^t u(C_t), \qquad 0 < \beta < 1.
\]
The household supplies $\bar{L}$ units of unskilled labor and $\bar{H}$ units of skilled labor, where $\bar{L} + \bar{H} = 1$.

The household's budget constraint is:
\begin{equation}
C_t + B_{t+1} = w_L \bar{L} + w_H \bar{H} + (1+r) B_t + \Pi_t,
\label{eq:hh_budget}
\end{equation}
where $B_t$ denotes household deposits at banks and $\Pi_t$ is aggregate firm dividends net of entry costs (i.e., $\Pi_t$ includes the value distributed by exiting firms minus the equity injected into entrants).

The household's Euler equation with linear utility pins down:
\begin{equation}
1 + r = \frac{1}{\beta}.
\label{eq:interest_rate}
\end{equation}

\subsubsection{Financial Intermediation}
Competitive banks accept deposits from households at rate $r$ and lend to firms at the same rate. Banks perfectly enforce repayment up to collateral value and make zero profits. No default occurs in equilibrium---the collateral constraint binds ex ante, preventing default, following \citet{kiyotaki1997credit}.

\subsection{Technology}

\subsubsection{Idiosyncratic Productivity}
\[
\log z_{j,t+1} = \rho_z \log z_{j,t} + \sigma_z \varepsilon_{j,t+1}, \quad \varepsilon_{j,t+1} \sim N(0,1),
\]
with $0 < \rho_z < 1$, $\sigma_z > 0$.

\subsubsection{Production Technology}

Production involves a nested CES structure. First, define the capital composite:
\begin{equation}
X_{j,t} = \Big[\theta_K K_{j,t}^{\rho_K} + \theta_Q Q_{j,t}^{\rho_K}\Big]^{1/\rho_K},
\label{eq:capital_composite}
\end{equation}
where $Q_{j,t}$ is an intangible-skill bundle defined as:
\begin{equation}
Q_{j,t} = \Big[\omega S_{j,t}^{\rho_Q} + (1-\omega)(H^P_{j,t})^{\rho_Q}\Big]^{1/\rho_Q}.
\label{eq:intangible_skill_bundle}
\end{equation}

The final production function is:
\begin{equation}
Y_{j,t} = z_{j,t} \Big[X_{j,t}^{\alpha} L_{j,t}^{\gamma}\Big]^{\nu},
\label{eq:prod}
\end{equation}
where $\alpha, \gamma, \nu \in (0,1)$, $\theta_K, \theta_Q, \omega \in (0,1)$, and $\alpha + \gamma = 1$ (constant returns in the inner nest).

\begin{assumption}[Complementarity Structure]
\label{ass:complementarity}
Define elasticities $\sigma_K \equiv 1/(1-\rho_K)$ and $\sigma_Q \equiv 1/(1-\rho_Q)$. We assume $\sigma_Q < \sigma_K$ (i.e., $\rho_Q < \rho_K < 0$), so that intangibles and skilled labor are stronger complements than tangible capital and the intangible-skill bundle.
\end{assumption}

\begin{remark}
Assumption \ref{ass:complementarity} captures the idea that intangible capital and skilled labor work together in a more integrated way than tangible capital does with the intangible-skill composite $Q$. Empirical evidence from \citet{gozenozkara2024} supports the existence of synergies between intangibles and skilled labor.
\end{remark}

\subsubsection{Capital Accumulation}

Tangible capital evolves via standard accumulation:
\begin{equation}
K_{j,t+1} = (1-\delta_K) K_{j,t} + I^K_{j,t},
\label{eq:capital_accum_K}
\end{equation}
where $I^K_{j,t}$ is tangible investment and $0 < \delta_K < 1$.

Intangible capital is produced via R\&D labor:
\begin{equation}
S_{j,t+1} = (1-\delta_S) S_{j,t} + \Gamma (H^R_{j,t})^{\xi}, \qquad 0 < \xi \le 1,
\label{eq:capital_accum_S}
\end{equation}
where $H^R_{j,t}$ is skilled labor allocated to R\&D, $\Gamma > 0$, and $\xi \le 1$ captures (weakly) decreasing returns to R\&D labor in knowledge creation.\footnote{The case $\xi < 1$ captures diminishing returns in R\&D production. The limiting case $\xi = 1$ corresponds to linear intangible production, as in \citet{atkeson2010innovation}.}

Total skilled labor hired by the firm satisfies:
\begin{equation}
H_{j,t} = H^P_{j,t} + H^R_{j,t}.
\label{eq:skilled_labor_total}
\end{equation}

\subsubsection{Adjustment Costs}

Following the investment literature \citep{cooper2006dynamics, khan2008idiosyncratic}, capital adjustment is subject to convex costs:
\begin{equation}
\Phi^K_{j,t} = \frac{\phi_K}{2} \frac{(I^K_{j,t})^2}{K_{j,t}}, \qquad
\Phi^S_{j,t} = \frac{\phi_S}{2} \frac{(\Delta S_{j,t})^2}{S_{j,t}},
\label{eq:adj_costs}
\end{equation}
where $\Delta S_{j,t} \equiv S_{j,t+1} - (1-\delta_S)S_{j,t} = \Gamma(H^R_{j,t})^{\xi}$ is gross intangible investment. Setting $\phi_K = \phi_S = 0$ nests the frictionless case.

\subsection{Financial Frictions}

\subsubsection{Collateral Constraint}

Following the costly-state-verification literature \citep{townsend1979optimal, bernanke1999financial}, lenders can recover only a fraction of firm assets in case of default. Define recovery rates:
\begin{equation}
\alpha_K \in (0,1), \qquad \alpha_S \in [0, \alpha_K),
\label{eq:recovery_rates}
\end{equation}
where $\alpha_K$ is the recovery rate on tangible capital and $\alpha_S$ is the recovery rate on intangible capital, with $\alpha_S < \alpha_K$ reflecting the lower pledgeability of intangibles \citep{holttinenmelolinnafroem2025}.

\begin{assumption}[Low Intangible Pledgeability]
\label{ass:pledgeability}
Intangible assets are less pledgeable than tangible assets: $0 \le \alpha_S < \alpha_K < 1$. The baseline calibration considers $\alpha_S = 0$ (intangibles fully non-pledgeable).
\end{assumption}

To ensure no default in equilibrium, banks lend only up to recoverable collateral value. Following the timing convention in \citet{kiyotaki1997credit} and \citet{jermann2012macroeconomic}, debt $D_{j,t+1}$ taken in period $t$ (repaid in $t+1$) is collateralized by capital available when repayment is due:
\begin{equation}
D_{j,t+1} \le \alpha_K K_{j,t+1} + \alpha_S S_{j,t+1}.
\label{eq:collateral_constraint}
\end{equation}
Since $K_{j,t+1}$ and $S_{j,t+1}$ depend on current investment decisions, this creates a direct link between investment and borrowing capacity within the period.

\subsubsection{Within-Period Timing}

The stages below describe the sequence of events within period $t$:

\begin{enumerate}[leftmargin=*]
\item \textbf{Enter with state:} Firm enters with $(K_t, S_t, D_t)$ and observes productivity $z_t$ drawn from the AR(1) process conditional on $z_{t-1}$.

\item \textbf{Static production:} The firm hires labor $(L_t, H^P_t)$ at wages $(w_L, w_H)$ and produces output $Y_t$. Gross profits are $\pi(z_t, K_t, S_t) = Y_t - w_L L_t - w_H H^P_t$.

\item \textbf{Exit shock:} With probability $\zeta$, the firm exits. The exit value is:
\begin{equation}
V^{exit}(z, K, S, D) = \max\big\{\pi(z, K, S) + (1-\delta_K)K + (1-\delta_S)S - (1+r)D,\; 0\big\}.
\label{eq:exit_value}
\end{equation}
Exiting firms collect gross profits, sell undepreciated capital at its depreciated value, repay debt, and distribute the residual (bounded below by zero under limited liability) to shareholders.

\item \textbf{Investment and financing (survivors):} With probability $(1-\zeta)$, the firm survives and chooses tangible investment $I^K_t$, R\&D labor $H^R_t$, and new debt $D_{t+1}$ subject to the collateral constraint \eqref{eq:collateral_constraint} and the non-negative dividend constraint.

\item \textbf{Dividend payment:} The firm pays dividends $\text{Div}_t \ge 0$ to shareholders.

\item \textbf{End of period:} Firm enters $t+1$ with state $(z_t, K_{t+1}, S_{t+1}, D_{t+1})$.
\end{enumerate}

\textbf{Timing and Wage Equality.} Both $H^P$ (production) and $H^R$ (R\&D) are skilled labor employed within period $t$ at wage $w_H$. The single wage clears the aggregate skilled labor market: $\int (H^P_j + H^R_j) d\Psi = \bar{H}$.

\subsubsection{Budget Constraint and Dividends}

Define gross profits from production:
\begin{equation}
\pi(z, K, S) \equiv \max_{L, H^P} \Big\{ Y(z, K, S, L, H^P) - w_L L - w_H H^P \Big\}.
\label{eq:gross_profits_def}
\end{equation}

Dividends distributed to shareholders are:
\begin{equation}
\text{Div} = \pi(z, K, S) - I^K - \Phi^K - \Phi^S - w_H H^R - (1+r) D + D'.
\label{eq:dividends}
\end{equation}

Firms cannot issue new equity:
\begin{equation}
\text{Div} \ge 0.
\label{eq:div_constraint}
\end{equation}

\subsection{Firm Problem}

\subsubsection{Bellman Equation}

The value function incorporates the exit shock explicitly:
\begin{equation}
V(z, K, S, D) = \zeta \, V^{exit}(z, K, S, D) \;+\; (1-\zeta) \max_{I^K, H^R, D'} \Big\{ \text{Div} + \beta \, \E\big[V(z', K', S', D') \mid z\big] \Big\},
\label{eq:value_function}
\end{equation}
subject to:
\begin{itemize}
\item Dividends: $\text{Div} = \pi(z, K, S) - I^K - \Phi^K - \Phi^S - w_H H^R - (1+r) D + D'$

\item Capital accumulation: $K' = (1-\delta_K) K + I^K$, \quad $S' = (1-\delta_S) S + \Gamma (H^R)^{\xi}$

\item Collateral constraint: $D' \le \alpha_K K' + \alpha_S S'$

\item Non-negative dividends: $\text{Div} \ge 0$

\item Limited liability on continuation: $\max\{\cdot, 0\}$ inside the $(1-\zeta)$ term
\end{itemize}

The exit value $V^{exit}$ depends only on the current state, not on investment choices. The continuation problem is solved by survivors who choose $(I^K, H^R, D')$.

\subsubsection{Analytical Determination of $D'$}

Following the logic of \citet{khan2008idiosyncratic}, the optimal debt choice $D'$ can be determined analytically for each $(I^K, H^R)$ combination. Define:
\[
\text{cash flow} \equiv \pi(z,K,S) - (1+r)D, \qquad \text{expenses} \equiv I^K + \Phi^K + \Phi^S + w_H H^R.
\]
The net financing need is $D_{\text{needed}} = \text{expenses} - \text{cash flow}$. Given that $\beta(1+r) = 1$ and exit risk makes each dollar of excess debt costly (expected loss $\zeta(1+r)$ per dollar), the firm never borrows more than necessary. The solution is:
\begin{itemize}
\item \textbf{Type A} ($\mu = 0, \lambda = 0$): If $D_{\text{needed}} \le 0$, set $D' = 0$ and $\text{Div} = -D_{\text{needed}} > 0$.
\item \textbf{Type B} ($\mu > 0, \lambda = 0$): If $0 < D_{\text{needed}} < \bar{D}$, set $D' = D_{\text{needed}}$ and $\text{Div} = 0$.
\item \textbf{Type C} ($\mu > 0, \lambda > 0$): If $D_{\text{needed}} = \bar{D} \equiv \alpha_K K' + \alpha_S S'$, the collateral constraint binds.
\item \textbf{Infeasible}: If $D_{\text{needed}} > \bar{D}$, this $(I^K, H^R)$ combination cannot be financed.
\end{itemize}
This eliminates one dimension from the numerical optimization, leaving a grid search over $(I^K, H^R)$ only.

\subsubsection{First-Order Conditions}

Let $\mu \ge 0$ denote the Lagrange multiplier on the dividend constraint \eqref{eq:div_constraint}, and $\lambda \ge 0$ the multiplier on the collateral constraint \eqref{eq:collateral_constraint}. When $\mu = 0$, dividends are positive and internal funds are not scarce; when $\mu > 0$, the dividend constraint binds and internal funds command a premium. When $\lambda = 0$, the collateral constraint is slack; when $\lambda > 0$, it binds.

\paragraph{FOC for Debt ($D'$).}
\begin{equation}
1 + \mu = \lambda + \beta(1-\zeta)(1+r) \E[(1 + \mu')].
\label{eq:foc_D}
\end{equation}
For an unconstrained firm ($\mu = \lambda = 0$), this reduces to $1 = \beta(1-\zeta)(1+r)$, which holds when exit risk exactly offsets the discount factor.

\paragraph{FOC for Tangible Investment ($I^K$).}
\begin{equation}
(1 + \mu)\left(1 + \phi_K \frac{I^K}{K}\right) = \lambda \alpha_K + \beta(1-\zeta) \E[V_{K'}].
\label{eq:foc_IK}
\end{equation}

The left-hand side is the marginal cost: purchasing one unit plus adjustment costs, valued at $(1+\mu)$ reflecting the shadow cost of internal funds. The right-hand side has two components:
\begin{enumerate}
\item $\lambda \alpha_K$: the immediate collateral benefit---investment increases $K'$, relaxing the current-period borrowing constraint.
\item $\beta(1-\zeta) \E[V_{K'}]$: the expected continuation value of additional capital.
\end{enumerate}

\paragraph{Envelope Condition for $K$.}
\begin{equation}
V_K = (1+\mu) \frac{\partial \pi}{\partial K} + (1+\mu) \frac{\phi_K}{2}\left(\frac{I^K}{K}\right)^2 + (1-\delta_K)\Big[\lambda \alpha_K + \beta(1-\zeta) \E[V_{K'}]\Big].
\label{eq:envelope_K}
\end{equation}

The term $(1+\mu) \frac{\phi_K}{2}(I^K/K)^2$ reflects that higher capital reduces adjustment costs for given investment. The collateral term $\lambda \alpha_K$ is multiplied by $(1-\delta_K)$ because only undepreciated capital carries forward.

Combining \eqref{eq:foc_IK} and \eqref{eq:envelope_K} yields the Euler equation:
\begin{align}
(1 + \mu_t)\left(1 + \phi_K \frac{I^K_t}{K_t}\right) &= \lambda_t \alpha_K + \beta(1-\zeta) \E_t\Bigg[ (1+\mu_{t+1}) \frac{\partial \pi_{t+1}}{\partial K_{t+1}} + (1+\mu_{t+1}) \frac{\phi_K}{2}\left(\frac{I^K_{t+1}}{K_{t+1}}\right)^2 \nonumber \\
&\quad + (1-\delta_K)\Big(\lambda_{t+1} \alpha_K + \beta(1-\zeta) \E_{t+1}[V_{K,t+2}]\Big)\Bigg].
\label{eq:euler_K}
\end{align}

\paragraph{FOC for Intangible Investment (via $H^R$).} Since $\partial S'/\partial H^R = \Gamma \xi (H^R)^{\xi-1}$:
\begin{equation}
(1+\mu)\left(w_H + \phi_S \frac{\Delta S}{S} \cdot \Gamma \xi (H^R)^{\xi-1}\right) = \Big[\lambda \alpha_S + \beta(1-\zeta) \E[V_{S'}]\Big] \cdot \Gamma \xi (H^R)^{\xi-1}.
\label{eq:foc_HR}
\end{equation}

Define the marginal cost of producing intangibles as $q_S \equiv w_H / [\Gamma \xi (H^R)^{\xi-1}]$. Abstracting from adjustment costs, the FOC simplifies to:
\begin{equation}
(1+\mu) \cdot q_S = \lambda \alpha_S + \beta(1-\zeta) \E[V_{S'}].
\label{eq:foc_S}
\end{equation}

\paragraph{Envelope Condition for $S$.}
\begin{equation}
V_S = (1+\mu) \frac{\partial \pi}{\partial S} + (1+\mu) \frac{\phi_S}{2}\left(\frac{\Delta S}{S}\right)^2 + (1-\delta_S)\Big[\lambda \alpha_S + \beta(1-\zeta) \E[V_{S'}]\Big].
\label{eq:envelope_S}
\end{equation}

\paragraph{Envelope Condition for $D$.}
\begin{equation}
V_D = -(1+\mu)(1+r).
\label{eq:envelope_D}
\end{equation}

\subsubsection{The Pecking-Order Distortion}

Comparing the FOCs for tangible \eqref{eq:foc_IK} and intangible \eqref{eq:foc_S} investment reveals the core mechanism.

\begin{proposition}[Pecking-Order Distortion]
\label{prop:pecking_order}
For a financially constrained firm ($\lambda > 0$), the immediate collateral benefit is larger for tangible than intangible investment:
\begin{equation}
\underbrace{\lambda \alpha_K}_{\text{collateral benefit of } K'} > \underbrace{\lambda \alpha_S}_{\text{collateral benefit of } S'}
\label{eq:pecking_order}
\end{equation}
since $\alpha_K > \alpha_S$.
\end{proposition}

\textbf{Intuition.} Each unit of tangible capital generates borrowing capacity $\alpha_K$, while intangible capital generates only $\alpha_S < \alpha_K$. When the collateral constraint binds ($\lambda > 0$), this asymmetry creates an implicit subsidy for tangible investment: the effective cost of tangible investment is reduced by $\lambda \alpha_K / (1+\mu)$, while intangible investment is reduced by only $\lambda \alpha_S / (1+\mu)$.

For unconstrained firms ($\lambda = 0$), the collateral terms vanish and investment depends only on marginal products---the first-best allocation.

\paragraph{Capital Composition Distortion.} Rearranging the FOCs in steady state:
\begin{equation}
\frac{\text{MB}_K - \lambda \alpha_K}{\text{MB}_S - \lambda \alpha_S} = \frac{(1+\mu)(1 + \phi_K I^K/K)}{(1+\mu) q_S},
\label{eq:capital_ratio}
\end{equation}
where $\text{MB}_K$ and $\text{MB}_S$ denote the continuation value terms. For constrained firms, since $\alpha_K > \alpha_S$, the numerator is reduced more than the denominator, distorting capital composition toward tangibles.

\subsection{Entry and Exit}

A mass $\zeta$ of firms exits each period. Upon exit, shareholders receive $V^{exit}$ as defined in \eqref{eq:exit_value}. An equal mass $\zeta$ of new firms enters to maintain unit mass. Both the exit rate and entry mass are exogenous. Entrants draw productivity $z_0$ from the stationary distribution $F_z$ and are endowed with capital equal to the cross-sectional average:
\begin{equation}
K_0 = \E_{\Psi^*}[K], \qquad S_0 = \E_{\Psi^*}[S], \qquad D_0 = 0,
\label{eq:entry_condition}
\end{equation}
where $\E_{\Psi^*}[\cdot]$ denotes the cross-sectional mean under the stationary distribution. The capital endowment $(K_0, S_0)$ and the stationary distribution $\Psi^*$ are jointly determined in a fixed point.

\subsection{Equilibrium}

\begin{definition}[Stationary Recursive Equilibrium]
\label{def:equilibrium}
A stationary recursive equilibrium consists of:
\begin{enumerate}[label=(\roman*)]
\item Firm value function $V(z, K, S, D)$ and policy functions $\{K'(\cdot), S'(\cdot), D'(\cdot), I^K(\cdot), H^R(\cdot), L(\cdot), H^P(\cdot)\}$,
\item Wages $(w_L, w_H)$ and interest rate $r$,
\item A stationary distribution of firms $\Psi^*(z, K, S, D)$,
\item Entry capital $(K_0, S_0) = (\E_{\Psi^*}[K], \E_{\Psi^*}[S])$,
\item Aggregate quantities $\{C, Y_{agg}, K_{agg}, S_{agg}, D_{agg}\}$,
\end{enumerate}
such that:
\begin{enumerate}[label=(\alph*)]
\item \textbf{Firms optimize:} $V(z, K, S, D)$ solves \eqref{eq:value_function} and policies satisfy the FOCs.

\item \textbf{Household optimizes:} $1 + r = 1/\beta$ from the Euler equation.

\item \textbf{Labor markets clear:}
\[
\int L_j \, d\Psi^* = \bar{L}, \qquad \int (H^P_j + H^R_j) \, d\Psi^* = \bar{H}.
\]

\item \textbf{Credit market clears:}
\[
B = \int D_j \, d\Psi^*.
\]

\item \textbf{Goods market clears:}
\[
C + \int \big(I^K_j + \Phi^K_j + \Phi^S_j\big) d\Psi^* = \int Y_j \, d\Psi^* + \zeta \int \big[(1-\delta_K)K_j + (1-\delta_S)S_j\big] d\Psi^* - \zeta (K_0 + S_0).
\]
The right-hand side adds output, the depreciated capital recovered from exiting firms, and subtracts the capital endowed to entrants.

\item \textbf{Entry condition:} Entrants start at $(K_0, S_0, 0) = (\E_{\Psi^*}[K], \E_{\Psi^*}[S], 0)$.

\item \textbf{Stationarity:} $\Psi^*$ is invariant under the transition implied by policy functions, exit, and entry.
\end{enumerate}
\end{definition}
