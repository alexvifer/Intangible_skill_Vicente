\documentclass[12pt]{article}
\usepackage[a4paper, margin=1in]{geometry}
\usepackage{amsmath, amsfonts, amssymb, amsthm, bm}
\usepackage{graphicx, caption, subcaption, float, booktabs}
\usepackage{enumitem, hyperref, natbib, xcolor, dsfont, scrextend, authblk}
\usepackage[bottom]{footmisc}

\deffootnote{1.5em}{0em}{\thefootnotemark\quad}
\renewcommand\thesection{\Roman{section}.}
\renewcommand\thesubsection{\thesection\Alph{subsection}.}
\renewcommand\thesubsubsection{\arabic{subsubsection}.}

\hypersetup{
  colorlinks=true,
  linkcolor=blue,
  citecolor=blue,
  urlcolor=cyan
}

\newtheorem{proposition}{Proposition}
\newtheorem{hypothesis}{Hypothesis}
\newtheorem{lemma}{Lemma}
\newtheorem{theorem}{Theorem}
\newtheorem{definition}{Definition}
\newtheorem{corollary}{Corollary}
\newtheorem{assumption}{Assumption}
\newtheorem*{remark}{Remark}

\DeclareMathOperator*{\argmax}{arg\,max}
\DeclareMathOperator*{\E}{\mathbb{E}}

% Figure note command
\newcommand{\note}[1]{\par\vspace{6pt}\footnotesize\textit{Note:} #1}

\title{\textbf{Financial Frictions, Intangible Capital and Productivity: A Model of Skill-Biased Stagnation}}
\author[1]{Alejandro Vicente}
\affil[1]{University of Alicante}
\date{\textcolor{red}{PRELIMINARY AND INCOMPLETE, DO NOT CIRCULATE} \\ \today}

\begin{document}
\maketitle

\begin{abstract}
This paper studies how financial frictions prevent economies from fully exploiting productivity gains from rising human capital. Using Portuguese matched employer-employee and firm balance sheet data for 2011-2022, I document three empirical facts: (1) intangible capital and skilled labor are complements in production, (2) financially constrained firms underinvest in intangibles and R\&D labor, and (3) constrained firms underexploit the complementarity: during Portugal's skill supply expansion, wage premia declined more sharply at constrained firms despite hiring skilled workers at similar rates. I develop a quantitative heterogeneous-firm model with nested CES production, endogenous intangible accumulation through R\&D, and collateral constraints with differential pledgeability (\(\alpha_K > \alpha_S\)). The model rationalizes the empirical patterns and reveals a \emph{skill-biased stagnation} mechanism: when skilled labor becomes more abundant, unconstrained firms exploit the complementarity by investing in both intangibles and skills, while constrained firms hire skilled workers but cannot finance complementary intangible investments, leading to underexploitation and muted aggregate productivity gains. Counterfactual experiments quantify the role of financial frictions in limiting returns to human capital accumulation.
\end{abstract}

\section{Introduction}

Over the past decades, advanced economies have experienced substantial increases in educational attainment alongside rising importance of intangible capital in production. Yet aggregate productivity growth has been disappointing. This paper proposes a mechanism through which financial frictions can prevent economies from fully exploiting the potential gains from higher human capital: \emph{skill-biased stagnation}.

The mechanism operates through three key channels. First, intangible capital and skilled labor are complements in production, implying that firms with high intangible intensity benefit more from skilled workers. Second, intangibles are harder to pledge as collateral than tangible assets, creating a pecking-order distortion where financially constrained firms underinvest in intangibles. Third, when skilled labor becomes more abundant, unconstrained firms exploit the complementarity by investing in both intangibles and skilled workers, while constrained firms cannot finance the necessary intangible investments and therefore underexploit their skilled labor.

I develop this argument in two stages. Using Portuguese matched employer-employee and balance sheet data for 2011-2022, I establish three empirical facts: (1) intangible intensity and skilled labor share are complements in production, (2) financially constrained firms invest less in intangibles and R\&D labor, and (3) the complementarity between intangibles and skills is significantly weaker for constrained firms, with dynamic evidence showing that wage premia declined more sharply at constrained firms during the skill supply expansion. I then build a quantitative heterogeneous-firm model with nested CES production technology, endogenous intangible capital accumulation through R\&D, and collateral constraints with differential pledgeability. The model rationalizes the empirical patterns and enables counterfactual analysis of skill supply shocks under different financial friction scenarios.

The paper contributes to three literatures. First, it connects to work on intangible capital and firm dynamics \citep{peterstaylor2017,atkeson2010innovation}, emphasizing the role of complementarities with skilled labor. Second, it relates to the literature on financial frictions and capital misallocation \citep{buera2013finance,moll2014aggregate}, highlighting the specific distortions created by differential pledgeability of capital types. Third, it speaks to the productivity slowdown debate by proposing financial frictions as a mechanism limiting the returns to human capital accumulation.

The remainder of the paper is organized as follows. Section \ref{sec:data_evidence} presents the data and empirical evidence. Section \ref{sec:toy_model} develops intuition with a stylized two-period model. Section \ref{sec:quantitative_model} presents the full quantitative model. Section \ref{sec:plan} outlines the quantitative experiments. The Appendix contains details on data construction and cleaning procedures.

\section{Data and Empirical Evidence}
\label{sec:data_evidence}

\subsection{Data Sources and Measurement}

The empirical analysis uses Portuguese firm-level data from two administrative sources for 2011-2022. The \textbf{SCIE (Sistema de Contas Integradas das Empresas)} provides balance sheet and income statement data for all Portuguese firms required to file annual accounts, including assets, liabilities, revenue, costs, and R\&D expenditures. The \textbf{QP (Quadros de Pessoal)} is a matched employer-employee dataset containing worker-level information that I aggregate to construct firm-level skill composition measures.

I define skilled workers as those with tertiary education (ISCED 5-8) and construct the share of skilled workers as the key skill measure. Intangible capital stocks are built using the perpetual inventory method following a simplified version of the approach by \citet{peterstaylor2017}, combining knowledge capital from R\&D expenditures and balance sheet intangibles. All monetary variables are deflated to constant 2020 prices. After cleaning and sample restrictions (positive employment, non-negative balance sheet items, at least two consecutive observations per firm), the final analytical sample contains 1,759,093 firm-year observations. Details on data construction and cleaning are provided in Appendix \ref{app:data_cleaning}.

Financial constraints are proxied by leverage (total debt relative to total assets). Following standard practice in the corporate finance literature, I classify firms as constrained if their leverage exceeds the sector-year median, allowing for industry heterogeneity in optimal capital structure.

\subsection{Empirical Facts}

\subsubsection*{Fact 1: Complementarity Between Intangibles and Skilled Labor}

To test for complementarity between intangibles and skilled labor, I estimate the following production function specification:
\begin{equation}
\begin{aligned}
\ln Y_{jt} =\;& \beta_1 \, \text{IntangIntensity}_{jt}
+ \beta_2 \, \text{ShareSkilled}_{jt}
+ \beta_3 \, (\text{IntangIntensity} \times \text{ShareSkilled})_{jt} \\
& + \bm{X}'_{jt}\bm{\gamma}
+ \alpha_j + \delta_t + \eta_i + \varepsilon_{jt},
\end{aligned}
\label{eq:complementarity}
\end{equation}

where $Y_{jt}$ is gross value added for firm $j$ in year $t$, $\text{IntangIntensity}_{jt} \equiv K^{intang}_{jt}/K^{total}_{jt}$ is the ratio of intangible to total capital, $\text{ShareSkilled}_{jt}$ is the share of workers with tertiary education, $\bm{X}_{jt}$ includes log total capital, log employment, and log firm age, and $\alpha_j$, $\delta_t$, $\eta_i$ denote firm, year, and industry fixed effects respectively. The coefficient of interest is $\beta_3$: a positive estimate indicates that the marginal product of skilled labor is increasing in intangible intensity, i.e., complementarity.

Table \ref{tab:complementarity} reports estimates of equation \eqref{eq:complementarity}. The table presents a nested specification: column (1) includes no controls or fixed effects, column (2) adds firm, year, and industry fixed effects, and column (3) further includes the control variables $\bm{X}_{jt}$.

\begin{table}[htbp]\centering
\def\sym#1{\ifmmode^{#1}\else\(^{#1}\)\fi}
\caption{Complementarity Between Intangibles and Skilled Labor\label{tab:complementarity}}
\begin{tabular}{l*{3}{c}}
\toprule
            &\multicolumn{1}{c}{(1)}&\multicolumn{1}{c}{(2)}&\multicolumn{1}{c}{(3)}\\
            &\multicolumn{1}{c}{Revenue}&\multicolumn{1}{c}{Production}&\multicolumn{1}{c}{GVA}\\
\midrule
Intangible Intensity&     -0.1018\sym{***}&     -0.1111\sym{***}&     -0.1254\sym{***}\\
            &    (0.0065)         &    (0.0061)         &    (0.0089)         \\
\addlinespace
Share Skilled Workers&     -0.0128\sym{***}&     -0.0084\sym{**} &     -0.0154\sym{***}\\
            &    (0.0036)         &    (0.0033)         &    (0.0046)         \\
\addlinespace
Intangible Intensity $\times$ Share Skilled&      0.0262\sym{*}  &      0.0417\sym{***}&      0.0398\sym{**} \\
            &    (0.0137)         &    (0.0125)         &    (0.0175)         \\
\midrule
Observations&   1,759,084         &   1,759,084         &   1,759,076         \\
Adjusted R-squared&       0.952         &       0.951         &       0.895         \\
\bottomrule
\multicolumn{4}{l}{\footnotesize All output measures in logs. Intangible intensity = K{intangible}$ / K{total}$.}\\
\multicolumn{4}{l}{\footnotesize All specifications include firm, year, and industry fixed effects.}\\
\multicolumn{4}{l}{\footnotesize Robust standard errors in parentheses.}\\
\end{tabular}
\end{table}


The interaction coefficient is positive and statistically significant across all specifications, demonstrating robust complementarity between intangibles and skilled labor. The coefficient attenuates from 0.111 without controls to 0.040 with the full set of fixed effects and controls, indicating that the raw correlation partly reflects selection (intangible-intensive firms hiring more skilled workers) but a substantial within-firm complementarity remains. This pattern indicates that the marginal product of skilled labor is increasing in intangible intensity: firms with higher intangible capital benefit more from employing skilled workers. Conversely, the marginal product of intangibles is increasing in skill share. This complementarity is precisely the technological structure embedded in the nested CES production function of the quantitative model. Table \ref{tab:complementarityrobust} in Appendix \ref{app:robustness} shows that results are robust to using revenue or production value as alternative outcome measures.

Figure \ref{fig:fact1_correlation} shows the raw correlation between intangible intensity and skill share using a binscatter with 20 quantiles. The positive relationship is evident even without controlling for firm characteristics, though the regression estimates exploit within-firm variation and control for confounding factors.

\begin{figure}[htbp]
\centering
\includegraphics[width=0.7\textwidth]{QP_SCIE/results/fact1_raw_correlation.png}
\caption{Raw Correlation Between Intangible Intensity and Skilled Labor Share}
\label{fig:fact1_correlation}
\end{figure}

\subsubsection*{Fact 2: Financial Constraints and the Pecking Order}

The model predicts that financially constrained firms underinvest in intangibles relative to tangibles because intangibles have lower pledgeability. Figure \ref{fig:fact2_pecking_order} presents residualized binscatters showing the relationship between leverage (the constraint proxy) and various investment decisions, after partialling out firm, year, and industry fixed effects along with log capital, employment, and age.

\begin{figure}[htbp]
\centering
\begin{subfigure}[b]{0.48\textwidth}
\includegraphics[width=\textwidth]{QP_SCIE/results/fact2_physical_inv_leverage.png}
\caption{Physical Investment Rate}
\end{subfigure}
\hfill
\begin{subfigure}[b]{0.48\textwidth}
\includegraphics[width=\textwidth]{QP_SCIE/results/fact2_intangible_inv_leverage.png}
\caption{Intangible Investment Rate}
\end{subfigure}

\vspace{0.5cm}

\begin{subfigure}[b]{0.48\textwidth}
\includegraphics[width=\textwidth]{QP_SCIE/results/fact2_intang_intensity_leverage.png}
\caption{Intangible Intensity}
\end{subfigure}
\hfill
\begin{subfigure}[b]{0.48\textwidth}
\includegraphics[width=\textwidth]{QP_SCIE/results/fact2_rd_workers_leverage.png}
\caption{Share of R\&D Workers}
\end{subfigure}

\caption{Financial Constraints and the Pecking Order Distortion}
\label{fig:fact2_pecking_order}
\note{All variables residualized by regressing on firm, year, and industry fixed effects, log total capital, log employment, and log firm age. Binscatters use 20 quantiles. Panel (d) restricts to firms with positive R\&D employment.}
\end{figure}

The evidence reveals a clear pecking-order pattern. While physical investment shows little systematic relationship with leverage (Panel a), intangible investment, intangible intensity, and R\&D worker allocation all decline sharply with leverage (Panels b-d). Constrained firms shift their investment composition toward tangible assets and allocate fewer skilled workers to R\&D activities, precisely as the model predicts when intangibles have lower collateral value than tangibles.

\subsubsection*{Fact 3: Underexploitation of Complementarity by Constrained Firms}

If financial constraints prevent firms from building intangible capital, they should also reduce the extent to which firms can exploit the technological complementarity between intangibles and skills. Table \ref{tab:underexploitation} tests this prediction by estimating equation \eqref{eq:complementarity} separately for the pooled sample (column 1), low-leverage firms (column 2), and high-leverage firms (column 3), where leverage groups are defined relative to the sector-year median.

\begin{table}[htbp]\centering
\def\sym#1{\ifmmode^{#1}\else\(^{#1}\)\fi}
\caption{Underexploitation of Intangibles-Skills Complementarity\label{tab:underexploitation}}
\begin{tabular}{l*{3}{c}}
\toprule
            &\multicolumn{3}{c}{Log Gross Value Added}                        \\\cmidrule(lr){2-4}
            &\multicolumn{1}{c}{All Firms}&\multicolumn{1}{c}{Low Leverage}&\multicolumn{1}{c}{High Leverage}\\
\midrule
\textit{Main effects:}&                     &                     &                     \\
\addlinespace
Intangible Intensity&      -0.13\sym{***}&      -0.13\sym{***}&      -0.12\sym{***}\\
            &     (0.01)         &     (0.01)         &     (0.01)         \\
\addlinespace
Share Skilled Workers&      -0.02\sym{***}&      -0.02\sym{***}&      -0.01         \\
            &     (0.00)         &     (0.01)         &     (0.01)         \\
\addlinespace
\textit{Complementarity:}&                     &                     &                     \\
\addlinespace
Intangible Intensity $\times$ Share Skilled&       0.04\sym{**} &       0.06\sym{**} &      -0.01         \\
            &     (0.02)         &     (0.03)         &     (0.03)         \\
\midrule
Observations&   1,759,076         &     852,524         &     856,239         \\
Adjusted R-squared&       0.895         &       0.904         &       0.906         \\
\bottomrule
\multicolumn{4}{l}{\footnotesize Dependent variable: Log gross value added (GVA).}\\
\multicolumn{4}{l}{\footnotesize Low- and high-leverage defined relative to sector-year median leverage.}\\
\multicolumn{4}{l}{\footnotesize All specifications include firm, year, and industry fixed effects.}\\
\multicolumn{4}{l}{\footnotesize Controls include log total capital, log employment, and log firm age.}\\
\multicolumn{4}{l}{\footnotesize Robust standard errors in parentheses.}\\
\end{tabular}
\end{table}


The interaction coefficient is 0.055 and significant for low-leverage firms (column 2), but statistically indistinguishable from zero ($-0.005$, p $>$ 0.10) for high-leverage firms (column 3). This stark difference indicates that constrained firms cannot exploit the complementarity: even when they employ skilled workers, the lack of complementary intangible capital limits productivity gains. The pooled specification (column 1) masks this heterogeneity, showing an intermediate coefficient of 0.040. Table \ref{tab:underexploitationrobust} in Appendix \ref{app:robustness} confirms that this pattern holds when using revenue or production value as alternative outcome measures.

\paragraph{Dynamic Evidence: Wage Premium Decline.} The cross-sectional evidence establishes underexploitation at a point in time. I now examine the dynamic implications over the 2011-2022 period, during which Portugal experienced a substantial skill supply expansion. Figure \ref{fig:fact3_aggregate_trends} plots the evolution of the skill share and wage premium. The skill share increased substantially, yet the wage premium declined, a puzzle for standard capital-skill complementarity models where skill supply increases induce capital deepening that sustains premia.

\begin{figure}[htbp]
\centering
\includegraphics[width=0.72\textwidth]{QP_SCIE/results/fact3_skill_premium_trends.png}
\caption{Skill Supply Expansion and Declining Wage Premium}
\label{fig:fact3_aggregate_trends}
\note{Averages across all firms. Skill share is the fraction of workers with tertiary education. Wage premium is the ratio of average skilled to unskilled wages within firms.}
\end{figure}

Figure \ref{fig:fact3_leverage_trends} decomposes this pattern by financial constraints. Panel (a) shows that both low- and high-leverage firms increased their skilled labor shares over the period at remarkably similar rates. However, Panel (b) reveals that the wage premium decline was much sharper at high-leverage (constrained) firms. This differential trend reveals the mechanism: constrained firms hire skilled workers at similar rates to unconstrained firms, but cannot finance the complementary intangible investments needed to maintain skilled workers' productivity, causing the wage premium to compress more sharply.

\begin{figure}[htbp]
\centering
\begin{subfigure}[b]{0.48\textwidth}
\includegraphics[width=\textwidth]{QP_SCIE/results/fact3_skills_by_leverage.png}
\caption{Skill Share by Leverage}
\end{subfigure}
\hfill
\begin{subfigure}[b]{0.48\textwidth}
\includegraphics[width=\textwidth]{QP_SCIE/results/fact3_premium_by_leverage.png}
\caption{Wage Premium by Leverage}
\end{subfigure}
\caption{Skill Adoption vs.\ Wage Premium by Financial Constraints}
\label{fig:fact3_leverage_trends}
\note{Averages by leverage group. Low/high leverage defined relative to sector-year median. Both panels show residualized values, controlling for firm size (log capital, log employment), age, and industry fixed effects. Constrained firms hire skilled workers at similar rates as unconstrained firms (Panel a), yet experience sharper wage premium declines (Panel b), consistent with underexploitation of the intangible-skill complementarity.}
\end{figure}

This dynamic evidence completes the empirical case for skill-biased stagnation. Constrained firms respond to the skill supply shock by hiring skilled workers, but their inability to build complementary intangible capital means these workers are underexploited. The wage premium falls more sharply at constrained firms, revealing that financial frictions prevent the economy from fully exploiting productivity gains from rising human capital.

\subsection{Summary of Empirical Evidence}

The three facts establish the key building blocks of the theoretical mechanism. First, technology features complementarity between intangibles and skills. Second, financial frictions distort investment composition toward tangibles and away from intangibles. Third, the combination of these forces prevents constrained firms from exploiting complementarity in both the cross-section and over time: the interaction coefficient is weak for constrained firms, and wage premia declined more sharply at constrained firms during the skill supply expansion of 2011-2022. The following sections develop a model that rationalizes these patterns and enables quantitative counterfactual analysis.

\section{Building Intuition: A Toy Model}
\label{sec:toy_model}

\subsection{Environment}

Consider a two-period economy with a representative firm and no aggregate uncertainty. The firm is endowed with initial tangible capital $K_0 > 0$, initial wealth $W_0 \geq 0$, and a lifetime allocation of skilled labor $\bar{H} > 0$. Productivity $z > 0$ is constant and known.

\textbf{Period 1 (Investment Phase):} The firm chooses tangible investment $I^K \geq 0$ and allocates skilled labor between R\&D ($H^R \geq 0$) and production for period 2 ($H^P = \bar{H} - H^R$). Intangible capital is produced linearly:
\begin{equation}
S = \Gamma H^R, \quad \Gamma > 0.
\label{eq:toy_intangible_prod}
\end{equation}

\textbf{Period 2 (Production Phase):} The firm produces output using accumulated capital and the skilled labor allocated in period 1. No further investment decisions are made.

\subsection{Technology}

\textbf{Capital Stocks.} Period 2 tangible capital is:
\begin{equation}
K = K_0 + I^K.
\end{equation}

\textbf{Intangible-Skill Composite.} The intangible capital $S$ from period 1 combines with the allocated production labor $H^P = \bar{H} - H^R$ to form:
\begin{equation}
Q(S, H^P) = \left[\omega S^{\rho} + (1-\omega)(H^P)^{\rho}\right]^{1/\rho}, \quad \omega \in (0,1), \, \rho < 0.
\label{eq:toy_Q}
\end{equation}
The elasticity of substitution $\sigma_Q = 1/(1-\rho) < 1$ captures complementarity.

\textbf{Production.} Period 2 output is:
\begin{equation}
Y = z K^{\alpha} Q(S, H^P)^{1-\alpha}, \quad \alpha \in (0,1).
\label{eq:toy_production}
\end{equation}

For tractability, I abstract from unskilled labor and normalize away depreciation.

\subsection{Financial Constraint}

The firm may borrow only against tangible assets. At the end of period 1, the pledgeable capital stock equals $K_0 + I^K$. Thus:
\begin{equation}
D \leq \alpha_K (K_0 + I^K), \quad \alpha_K \in (0,1),
\label{eq:toy_collateral}
\end{equation}
where $D$ denotes borrowing in period 1. Intangible capital is non-pledgeable: $\alpha_S = 0$.

\textbf{Period 1 Budget Constraint:}
\begin{equation}
I^K + w_H H^R \leq W_0 + D.
\label{eq:toy_budget_raw}
\end{equation}

If the constraint binds, $D = \alpha_K(K_0 + I^K)$, so:
\begin{align}
I^K + w_H H^R &\leq W_0 + \alpha_K(K_0 + I^K), \\
I^K(1 - \alpha_K) + w_H H^R &\leq W_0 + \alpha_K K_0.
\label{eq:toy_budget}
\end{align}

Define \emph{effective wealth}:
\[
W'_0 \equiv W_0 + \alpha_K K_0,
\]
so the constraint becomes:
\begin{equation}
I^K(1 - \alpha_K) + w_H H^R \leq W'_0.
\end{equation}

\textbf{Interpretation.} Each additional unit of tangible investment increases borrowing capacity by $\alpha_K$, effectively reducing its cost to $(1-\alpha_K)$. R\&D yields no such benefit. The firm's effective resources include its financial wealth $W_0$ plus the borrowing capacity from its initial capital stock ($\alpha_K K_0$).

\subsection{Firm Problem}

The firm chooses $(I^K, H^R)$ to maximize net output:
\begin{equation}
\max_{I^K \geq 0, \, H^R \in [0, \bar{H}]} \quad Y - I^K - w_H \bar{H}
\label{eq:toy_objective}
\end{equation}
subject to the budget constraint \eqref{eq:toy_budget}, where:
\begin{align}
Y &= z(K_0 + I^K)^{\alpha} Q(\Gamma H^R, \bar{H} - H^R)^{1-\alpha}, \\
Q(\Gamma H^R, \bar{H} - H^R) &= \left[\omega(\Gamma H^R)^{\rho} + (1-\omega)(\bar{H} - H^R)^{\rho}\right]^{1/\rho}.
\end{align}

Note that total skilled labor cost $w_H \bar{H}$ is fixed, so the firm's choice concerns only the allocation of $\bar{H}$ between R\&D and production.

\subsubsection{Unconstrained Solution}

When $W'_0$ is sufficiently large that constraint \eqref{eq:toy_budget} does not bind, the first-order conditions are:
\begin{align}
\frac{\partial Y}{\partial I^K} &= \alpha z K^{\alpha - 1} Q^{1-\alpha} = 1, \label{eq:toy_foc_K_uncon} \\
\frac{\partial Y}{\partial H^R} &= (1-\alpha) z K^{\alpha} Q^{-\alpha} \left[\frac{\partial Q}{\partial S} \Gamma - \frac{\partial Q}{\partial H^P}\right] = 0. \label{eq:toy_foc_HR_uncon}
\end{align}

From \eqref{eq:toy_foc_HR_uncon}, the optimal allocation balances the marginal product of intangibles (via R\&D) against the marginal product of direct production labor:
\begin{equation}
\frac{\partial Q}{\partial S} \Gamma = \frac{\partial Q}{\partial H^P}.
\label{eq:toy_balance}
\end{equation}

This is an interior solution exploiting the complementarity structure \eqref{eq:toy_Q}.

\subsubsection{Constrained Solution}

When constraint \eqref{eq:toy_budget} binds, form the Lagrangian:
\begin{equation}
\mathcal{L} = Y - I^K - w_H \bar{H} - \mu[I^K(1-\alpha_K) + w_H H^R - W'_0],
\end{equation}
where $\mu > 0$ is the shadow value of resources.

The first-order conditions become:
\begin{align}
\frac{\partial Y}{\partial I^K} &= 1 + \mu(1 - \alpha_K), \label{eq:toy_foc_K_con} \\
\frac{\partial Y}{\partial H^R} &= \mu w_H. \label{eq:toy_foc_HR_con}
\end{align}

\textbf{Key Observation.} Comparing \eqref{eq:toy_foc_K_uncon} with \eqref{eq:toy_foc_K_con}: the constrained firm requires a higher marginal product of capital, implying lower $I^K$.

Comparing \eqref{eq:toy_foc_HR_uncon} with \eqref{eq:toy_foc_HR_con}: the constrained firm requires a positive marginal product of R\&D labor (instead of zero), implying lower $H^R$ (and correspondingly higher $H^P$) than the unconstrained firm.

\subsection{Pecking-Order Distortion}

Before stating the main proposition, I establish that the constrained firm produces a lower intangible-skill composite.

\begin{lemma}[Suboptimal Allocation]
\label{lem:toy_Q_lower}
The constrained firm produces a lower intangible-skill composite than the unconstrained firm: $Q_c < Q_u$.
\end{lemma}

\begin{proof}
Define $\tilde{Q}(H^R; \bar{H}) \equiv Q(\Gamma H^R, \bar{H} - H^R)$ as the intangible-skill composite viewed as a function of the R\&D allocation for given $\bar{H}$. The unconstrained firm chooses $H^R_u$ to satisfy \eqref{eq:toy_balance}, which is equivalent to maximizing $\tilde{Q}$ over $H^R \in [0, \bar{H}]$:
\[
H^R_u = \argmax_{H^R \in [0,\bar{H}]} \tilde{Q}(H^R; \bar{H}).
\]
The constrained firm chooses $H^R_c < H^R_u$ (as established below in Proposition \ref{prop:toy_pecking_order}). Since $\tilde{Q}$ is strictly concave in $H^R$ and $H^R_c \neq H^R_u$, we have $\tilde{Q}(H^R_c; \bar{H}) < \tilde{Q}(H^R_u; \bar{H})$, i.e., $Q_c < Q_u$.
\end{proof}

\begin{remark}[Parameter Dependence and Limiting Cases]
\label{rem:limiting_cases}
The result $Q_c < Q_u$ and its magnitude depend critically on the elasticity parameter $\rho$. The proof requires strict concavity of $\tilde{Q}(H^R) = [\omega(\Gamma H^R)^{\rho} + (1-\omega)(\bar{H} - H^R)^{\rho}]^{1/\rho}$, which holds for $\rho < 1$ with an interior optimum. We examine the limiting cases:

\textbf{Case 1: $\rho \to -\infty$ (Leontief, $\sigma_Q \to 0$).} In this limit,
\[
Q \to \min\{S, H^P\}
\]
after appropriate normalization of $\omega$. The composite becomes $\tilde{Q}(H^R) = \min\{\Gamma H^R, \bar{H} - H^R\}$, which is maximized at the balanced allocation $H^R_u = \bar{H}/(1 + \Gamma)$, yielding $Q_u = \Gamma\bar{H}/(1+\Gamma)$. Any deviation creates complete bottlenecks: the constrained firm's underinvestment in $S$ implies $Q_c = S_c = \Gamma H^R_c < \Gamma H^R_u \leq Q_u$, with $H^P_c$ entirely wasted beyond the binding constraint. The distortion is maximally severe.

\textbf{Case 2: $\rho \to 0$ (Cobb-Douglas, $\sigma_Q \to 1$).} Taking the limit via L'H\^opital's rule,
\[
Q \to S^{\omega}(H^P)^{1-\omega}.
\]
The composite $\tilde{Q}(H^R) = \Gamma^{\omega}(H^R)^{\omega}(\bar{H} - H^R)^{1-\omega}$ is strictly concave with interior maximum at $H^R_u = \omega\bar{H}$. The result $Q_c < Q_u$ holds, with the gap determined by the curvature $\tilde{Q}''(H^R_u) = -\omega(1-\omega)\Gamma^{\omega}\bar{H}^{-1}(H^R_u)^{\omega-2}(\bar{H}-H^R_u)^{-1-\omega} < 0$.

\textbf{Case 3: $\rho \in (0,1)$ (Gross Substitutes, $\sigma_Q > 1$).} The CES function remains strictly concave for $\rho < 1$, so the analysis proceeds unchanged. The composite $\tilde{Q}$ has a unique interior maximum, and $Q_c < Q_u$ holds. However, because $S$ and $H^P$ are substitutes, the constrained firm can partially compensate for low $S$ by allocating more $H^P$, reducing the magnitude of $Q_u - Q_c$ relative to the complementary case.

\textbf{Case 4: $\rho \to 1$ (Perfect Substitutes, $\sigma_Q \to \infty$).} In this limit,
\[
Q \to \omega S + (1-\omega)H^P,
\]
and the composite becomes $\tilde{Q}(H^R) = \omega\Gamma H^R + (1-\omega)(\bar{H} - H^R) = [\omega\Gamma - (1-\omega)]H^R + (1-\omega)\bar{H}$. This is \emph{linear} in $H^R$, so $\tilde{Q}$ is no longer strictly concave. Three subcases arise:
\begin{itemize}
\item If $\omega\Gamma > 1-\omega$: corner solution $H^R_u = \bar{H}$ (all labor to R\&D).
\item If $\omega\Gamma < 1-\omega$: corner solution $H^R_u = 0$ (all labor to production).
\item If $\omega\Gamma = 1-\omega$: the firm is indifferent over all allocations, and $Q = (1-\omega)\bar{H}$ regardless of $H^R$.
\end{itemize}
In the knife-edge case $\omega\Gamma = 1-\omega$, the financial constraint creates no distortion in $Q$ (though it still distorts $K$). Otherwise, the constraint may or may not bind depending on which corner is optimal.

\textbf{Summary.} The baseline assumption $\rho < 0$ ensures: (i) strict concavity of $\tilde{Q}$ with interior optimum, (ii) the result $Q_c < Q_u$ for any constrained allocation, and (iii) economically significant complementarity that amplifies the distortion. As $\rho \to -\infty$, the distortion becomes arbitrarily severe; as $\rho \to 1$, the distortion vanishes in the knife-edge case or the problem degenerates to corner solutions.
\end{remark}

\begin{proposition}[Pecking Order]
\label{prop:toy_pecking_order}
Under the collateral constraint \eqref{eq:toy_collateral} with $\alpha_K \in (0,1)$ and $\alpha_S = 0$, constrained firms ($\mu > 0$) exhibit:

\begin{enumerate}[label=(\roman*)]
\item Lower R\&D investment: $H^R_c < H^R_u$
\item Lower intangible capital: $S_c < S_u$
\item Higher production labor: $H^P_c > H^P_u$
\item Lower tangible investment: $I^K_c < I^K_u$
\end{enumerate}

where subscripts $c$ and $u$ denote constrained and unconstrained firms respectively.
\end{proposition}

\begin{proof}
\textbf{Parts (i)-(iii):} From \eqref{eq:toy_foc_HR_con}, the constrained firm requires $\partial Y/\partial H^R = \mu w_H > 0$, while the unconstrained firm has $\partial Y/\partial H^R = 0$ at optimum. Since
\[
\frac{\partial Y}{\partial H^R} = (1-\alpha) z K^{\alpha} Q^{-\alpha} \left[\frac{\partial Q}{\partial S} \Gamma - \frac{\partial Q}{\partial H^P}\right],
\]
and the term $(1-\alpha) z K^{\alpha} Q^{-\alpha} > 0$, the constrained firm has
\[
\frac{\partial Q}{\partial S} \Gamma > \frac{\partial Q}{\partial H^P}
\]
at its optimum, whereas the unconstrained firm has equality. Computing the partial derivatives of \eqref{eq:toy_Q}:
\[
\frac{\partial Q}{\partial S} = \omega S^{\rho-1} Q^{1-\rho}, \qquad \frac{\partial Q}{\partial H^P} = (1-\omega)(H^P)^{\rho-1} Q^{1-\rho}.
\]
The condition $\frac{\partial Q}{\partial S} \Gamma > \frac{\partial Q}{\partial H^P}$ implies a lower $S/H^P$ ratio than the unconstrained optimum. Given the constraint $H^P = \bar{H} - H^R$ and $S = \Gamma H^R$, this requires $H^R_c < H^R_u$. Parts (ii) and (iii) follow immediately.

\textbf{Part (iv):} Define $f(I^K; Q) \equiv \alpha z (K_0 + I^K)^{\alpha-1} Q^{1-\alpha}$, which is decreasing in $I^K$ (since $\alpha - 1 < 0$) and increasing in $Q$.

The unconstrained firm satisfies $f(I^K_u; Q_u) = 1$.

For the constrained firm, evaluate $f$ at $I^K_u$ with $Q_c$:
\[
f(I^K_u; Q_c) = \alpha z (K_0 + I^K_u)^{\alpha-1} Q_c^{1-\alpha} < \alpha z (K_0 + I^K_u)^{\alpha-1} Q_u^{1-\alpha} = 1,
\]
where the inequality uses $Q_c < Q_u$ (Lemma \ref{lem:toy_Q_lower}) and $(1-\alpha) > 0$.

The constrained firm's FOC \eqref{eq:toy_foc_K_con} requires $f(I^K_c; Q_c) = 1 + \mu(1-\alpha_K) > 1$. Since $f(\cdot; Q_c)$ is strictly decreasing in $I^K$ and $f(I^K_u; Q_c) < 1 < f(I^K_c; Q_c)$, we must have $I^K_c < I^K_u$.
\end{proof}

\begin{remark}
The asymmetry arises from differential collateral value. Each unit of $I^K$ reduces the effective resource constraint by $(1-\alpha_K)$ because it generates $\alpha_K$ in borrowing capacity. In contrast, $H^R$ provides no collateral benefit since $\alpha_S = 0$. This creates an implicit subsidy for tangible investment relative to R\&D.
\end{remark}

\subsection{Underexploitation of Complementarity}

\begin{lemma}[Complementarity]
\label{lem:toy_complementarity}
The intangible-skill composite \eqref{eq:toy_Q} with $\rho < 0$ satisfies:
\begin{equation}
\frac{\partial^2 Q}{\partial S \, \partial H^P} = \omega(1-\omega)(1-\rho) S^{\rho-1} (H^P)^{\rho-1} Q^{1-2\rho} > 0.
\end{equation}
\end{lemma}

\begin{proof}
Direct differentiation of \eqref{eq:toy_Q}. All terms are positive when $\rho < 0$ since $(1-\rho) > 0$ and $(1-2\rho) > 0$.
\end{proof}

\begin{corollary}[Underexploitation of Skilled Labor]
\label{cor:toy_underexploitation}
Despite having higher production labor ($H^P_c > H^P_u$), the constrained firm has lower marginal product of that labor:
\begin{equation}
\frac{\partial Q}{\partial H^P}(S_c, H^P_c) < \frac{\partial Q}{\partial H^P}(S_u, H^P_u).
\end{equation}
\end{corollary}

\begin{proof}
The marginal product of production labor can be rewritten as:
\begin{equation}
\frac{\partial Q}{\partial H^P} = (1-\omega)(H^P)^{\rho-1} Q^{1-\rho} = (1-\omega) \left(\frac{Q}{H^P}\right)^{1-\rho}.
\label{eq:MP_HP_ratio}
\end{equation}
Since $\rho < 0$, we have $1 - \rho > 1$, so \eqref{eq:MP_HP_ratio} is strictly increasing in the ratio $Q/H^P$.

From Lemma \ref{lem:toy_Q_lower}, $Q_c < Q_u$. From Proposition \ref{prop:toy_pecking_order}(iii), $H^P_c > H^P_u$. Therefore:
\[
\frac{Q_c}{H^P_c} < \frac{Q_u}{H^P_u},
\]
which implies $\frac{\partial Q}{\partial H^P}(S_c, H^P_c) < \frac{\partial Q}{\partial H^P}(S_u, H^P_u)$.
\end{proof}

\textbf{Economic Interpretation.} The constrained firm overallocates skilled labor to production ($H^P$) relative to intangibles ($S$) because of the financial distortion. However, due to complementarity, the productivity of that labor is limited by the low stock of intangibles. This creates a \emph{double inefficiency}: wrong allocation \emph{and} underexploitation of the resources allocated to production.

\subsection{Skill-Biased Stagnation}

Consider a comparative static exercise: an increase in the skilled labor endowment $\bar{H}$.

\subsubsection{Unconstrained Firm Response}

For the unconstrained firm, the optimality condition \eqref{eq:toy_balance} determines the allocation of $\bar{H}$ between $H^R$ and $H^P$. Using the expressions for marginal products:
\[
\omega \Gamma (\Gamma H^R)^{\rho-1} = (1-\omega)(\bar{H} - H^R)^{\rho-1}.
\]
This condition pins down the \emph{ratio} $H^R/(\bar{H} - H^R)$ independently of $\bar{H}$. Therefore, the R\&D share $\phi^R \equiv H^R/\bar{H}$ is constant:
\begin{equation}
\frac{dH^R_u}{d\bar{H}} = \phi^R_u, \qquad \frac{dH^P_u}{d\bar{H}} = 1 - \phi^R_u.
\label{eq:unconstrained_response}
\end{equation}

Both margins expand proportionally, fully exploiting the complementarity.

\subsubsection{Constrained Firm Response}

The binding constraint implies:
\[
I^K_c(1-\alpha_K) + w_H H^R_c = W'_0.
\]
Differentiating:
\[
(1-\alpha_K)\,dI^K_c + w_H\, dH^R_c = 0.
\]

Because both constrained FOCs depend on 
\[
Q_c = Q(\Gamma H^R_c, \bar{H} - H^R_c),
\]
which varies with $\bar{H}$, the pair $(I^K_c, H^R_c)$ generally adjusts with $\bar{H}$.

However, the constraint imposes a strong wedge: additional skilled labor primarily enters production. Formally, the constrained R\&D response satisfies:
\begin{equation}
0 \le \frac{dH^R_c}{d\bar{H}} < \phi^R_u, \qquad \frac{dH^P_c}{d\bar{H}} = 1 - \frac{dH^R_c}{d\bar{H}}.
\label{eq:constrained_response}
\end{equation}

\begin{proposition}[Skill-Biased Stagnation]
\label{prop:toy_stagnation}
When the skilled labor endowment increases ($\bar{H} \uparrow$):
\begin{enumerate}[label=(\roman*)]
\item The unconstrained firm increases both $H^R$ and $H^P$ proportionally, building higher $S$ and exploiting complementarity.
\item The constrained firm expands $H^P$ strictly more than $H^R$:
\[
0 \le \frac{dH^R_c}{d\bar{H}} < \phi^R_u.
\]
\item The intangible-skill composite grows faster for the unconstrained firm:
\begin{equation}
\frac{dQ_u}{d\bar{H}} > \frac{dQ_c}{d\bar{H}} > 0.
\label{eq:dQ_comparison}
\end{equation}
\item The output gap between firms widens:
\begin{equation}
\frac{d(Y_u - Y_c)}{d\bar{H}} > 0.
\label{eq:output_gap_widens}
\end{equation}
\end{enumerate}
\end{proposition}

\begin{proof}
Parts (i)-(ii) follow from the derived responses.

\textbf{Part (iii):} By the envelope theorem, since the unconstrained firm maximizes $Q$ over the allocation of $\bar{H}$:
\[
\frac{dQ_u}{d\bar{H}} = \frac{\partial Q}{\partial H^P}\bigg|_{(S_u, H^P_u)}.
\]
For the constrained firm:
\[
\frac{dQ_c}{d\bar{H}} = \frac{\partial Q}{\partial S}\Gamma \frac{dH^R_c}{d\bar{H}} + \frac{\partial Q}{\partial H^P}\left(1 - \frac{dH^R_c}{d\bar{H}}\right).
\]
Since the unconstrained firm satisfies $\frac{\partial Q}{\partial S}\Gamma = \frac{\partial Q}{\partial H^P}$, we can write:
\[
\frac{dQ_u}{d\bar{H}} = \frac{\partial Q}{\partial H^P}\bigg|_{(S_u, H^P_u)} = \frac{\partial Q}{\partial S}\bigg|_{(S_u, H^P_u)}\Gamma = \left[\frac{\partial Q}{\partial S}\Gamma \phi^R_u + \frac{\partial Q}{\partial H^P}(1-\phi^R_u)\right]_{(S_u, H^P_u)}.
\]

From Corollary \ref{cor:toy_underexploitation}, $\frac{\partial Q}{\partial H^P}(S_u, H^P_u) > \frac{\partial Q}{\partial H^P}(S_c, H^P_c)$. Moreover, from the constrained firm's FOC, $\frac{\partial Q}{\partial S}(S_c, H^P_c)\Gamma > \frac{\partial Q}{\partial H^P}(S_c, H^P_c)$. Since $\frac{dH^R_c}{d\bar{H}} < \phi^R_u$ (part ii), the constrained firm places less weight on the higher-valued margin (R\&D) and more weight on the lower-valued margin (production), establishing \eqref{eq:dQ_comparison}.

\textbf{Part (iv):} From $Y = zK^{\alpha}Q^{1-\alpha}$:
\[
\frac{dY}{d\bar{H}} = (1-\alpha) z K^{\alpha} Q^{-\alpha} \frac{dQ}{d\bar{H}} = (1-\alpha) \frac{Y}{Q} \frac{dQ}{d\bar{H}}.
\]

Define the ``output-to-composite'' ratio $\psi \equiv Y/Q = zK^{\alpha}Q^{-\alpha}$. Then:
\[
\frac{d(Y_u - Y_c)}{d\bar{H}} = (1-\alpha)\left[\psi_u \frac{dQ_u}{d\bar{H}} - \psi_c \frac{dQ_c}{d\bar{H}}\right].
\]

From part (iii), $\frac{dQ_u}{d\bar{H}} > \frac{dQ_c}{d\bar{H}}$. Using \eqref{eq:MP_HP_ratio}:
\[
\frac{dQ}{d\bar{H}} \ge \frac{\partial Q}{\partial H^P} = (1-\omega)\left(\frac{Q}{H^P}\right)^{1-\rho}.
\]

Therefore:
\begin{align}
\frac{dY}{d\bar{H}} &\ge (1-\alpha) z K^{\alpha} Q^{-\alpha} (1-\omega)\left(\frac{Q}{H^P}\right)^{1-\rho} \nonumber \\
&= (1-\alpha)(1-\omega) z K^{\alpha} Q^{1-\alpha-\rho} (H^P)^{-(1-\rho)}.
\label{eq:dY_dH_expanded}
\end{align}

From Proposition \ref{prop:toy_pecking_order}: $K_u > K_c$, $Q_u > Q_c$, and $H^P_u < H^P_c$. Since $\alpha > 0$, $1-\alpha-\rho > 0$ (as $\rho < 0$), and $1-\rho > 0$:
\begin{itemize}
\item $K_u^{\alpha} > K_c^{\alpha}$
\item $Q_u^{1-\alpha-\rho} > Q_c^{1-\alpha-\rho}$
\item $(H^P_u)^{-(1-\rho)} > (H^P_c)^{-(1-\rho)}$
\end{itemize}
All three factors favor the unconstrained firm, so $\frac{dY_u}{d\bar{H}} > \frac{dY_c}{d\bar{H}}$, establishing \eqref{eq:output_gap_widens}.
\end{proof}

\textbf{Economic Intuition.} The skill-biased stagnation arises because constrained firms cannot finance the R\&D needed to build intangibles that would complement the additional skilled labor in production. While they can deploy more $H^P$, the lack of complementary $S$ means this labor is underexploited. Unconstrained firms, by contrast, expand both margins and realize the full gains from complementarity.

\newpage
\section{Quantitative Model}
\label{sec:quantitative_model}

\subsection{Environment}

\subsubsection{Time and Agents}
Time is discrete. A continuum of firms \(j\in[0,1]\) faces idiosyncratic productivity \(z_{j,t}\). Each holds tangible capital \(K_{j,t}\), intangible capital \(S_{j,t}\), and debt \(D_{j,t-1}\). Firms exit with probability \(\zeta\in(0,1)\); entrants pay entry cost \(c_e>0\) (financed by the household via lump-sum transfers), draw \(z_0\sim F_z\) on \([\underline z,\overline z]\), and start with \(K_0=S_0=D_{-1}=0\).

\subsubsection{Household}
A representative household owns all firms, supplies skilled and unskilled labor inelastically, and has standard preferences over consumption:
\[
\E_0\sum_{t=0}^{\infty}\beta^t u(C_t) , \qquad 0<\beta<1.
\]
The household supplies \(\bar L_t\) units of unskilled labor and \(\bar H_t\) units of skilled labor, where \(\bar L_t+\bar H_t=1\). 

The household's budget constraint is:
\begin{equation}
C_t + D_t^H = w_L \bar L_t + w_H \bar H_t + R D_{t-1}^H + \Pi_t - \zeta c_e N_t,
\label{eq:hh_budget}
\end{equation}
where \(D_t^H\) denotes household deposits at banks, \(\Pi_t\) aggregate firm profits (dividends), and \(N_t\) the mass of active firms. The term \(\zeta c_e N_t\) represents the consumption goods transferred to new entrants to cover entry costs, replacing exiting firms.

The household's Euler equation with linear utility pins down:
\begin{equation}
R = \frac{1}{\beta}.
\label{eq:interest_rate}
\end{equation}

\subsubsection{Financial Intermediation}
Competitive banks accept deposits from households at rate \(R\) and lend to firms. Banks perfectly enforce repayment up to collateral value and make zero profits. No default occurs in equilibrium—the collateral constraint binds ex ante, preventing default, following \citet{kiyotaki1997credit}.

\subsection{Technology}

\subsubsection{Idiosyncratic Productivity}
\[
\log z_{j,t+1}=\rho_z\log z_{j,t}+\sigma_z\varepsilon_{j,t+1},\quad 
\varepsilon_{j,t+1}\sim N(0,1),
\]
with \(0<\rho_z<1\), \(\sigma_z>0\).

\subsubsection{Production Technology}

Production involves a nested CES structure. First, define the capital composite:
\begin{equation}
X_{j,t} = \Big[\theta_K K_{j,t}^{\rho_K}+\theta_Q Q_{j,t}^{\rho_K}\Big]^{1/\rho_K},
\label{eq:capital_composite}
\end{equation}
where \(Q_{j,t}\) is an intangible-skill bundle defined as:
\begin{equation}
Q_{j,t}=\Big[\omega S_{j,t}^{\rho_Q}+(1-\omega)(H^P_{j,t})^{\rho_Q}\Big]^{1/\rho_Q}.
\label{eq:intangible_skill_bundle}
\end{equation}

The final production function is:
\begin{equation}
Y_{j,t}=z_{j,t}\Big[X_{j,t}^{\alpha}L_{j,t}^{\gamma}\Big]^{\nu},
\label{eq:prod}
\end{equation}
where \(\alpha,\gamma,\nu\in(0,1)\), \(\theta_K,\theta_Q,\omega\in(0,1)\), and \(\alpha+\gamma=1\) (constant returns in the inner nest).

\begin{assumption}[Complementarity Structure]
\label{ass:complementarity}
Define elasticities \(\sigma_K \equiv 1/(1-\rho_K)\) and \(\sigma_Q \equiv 1/(1-\rho_Q)\). We assume that \(\sigma_Q < \sigma_K\) (i.e., \(\rho_Q < \rho_K < 0\)), so that intangibles and skilled labor are stronger complements than tangible capital and the intangible-skill bundle.
\end{assumption}

\begin{remark}
Assumption \ref{ass:complementarity} captures the idea that intangible capital and skilled labor work together in a more integrated way than tangible capital does with the intangible-skill composite \(Q\). Empirical evidence from \citet{gozenozkara2024} supports the existence of sinergies between intangibles and skilled labor. 
\end{remark}

\subsubsection{Capital Accumulation}

Tangible capital stock evolves via standard time-to-build accumulation:
\begin{equation}
K_{j,t+1} = (1-\delta_K) K_{j,t} + I^K_{j,t},
\label{eq:capital_accum_K}
\end{equation}
where \(I^K_{j,t}\) is tangible investment and \(0<\delta_K<1\).

Intangible capital is produced via R\&D labor:
\begin{equation}
S_{j,t+1} = (1-\delta_S) S_{j,t} + \Gamma(H^R_{j,t})^{\xi}, \qquad 0<\xi\le 1,
\label{eq:capital_accum_S}
\end{equation}
where \(H^R_{j,t}\) is skilled labor allocated to R\&D, \(\Gamma>0\) and \(\xi\le 1\) captures (weakly) decreasing returns to R\&D labor in knowledge creation.\footnote{The case $\xi<1$ captures diminishing returns in R\&D production. The limiting case $\xi=1$ corresponds to linear intangible production, as in \citet{atkeson2010innovation} in the context of innovation. Extensions could incorporate knowledge spillovers by adding $S_{j,t}$ to the R\&D production function, creating path dependence in innovation capabilities.}

Total skilled labor satisfies:
\begin{equation}
H_{j,t} = H^P_{j,t} + H^R_{j,t}.
\label{eq:skilled_labor_total}
\end{equation}

\subsection{Financial Frictions}

\subsubsection{Collateral Constraint Microfoundation}

Following the costly-state-verification literature \citep{townsend1979optimal,bernanke1999financial}, we assume lenders can recover only a fraction of firm assets in case of default. Specifically, define recovery rates:
\begin{equation}
\alpha_K \in (0,1), \qquad \alpha_S \in [0, \alpha_K),
\label{eq:recovery_rates}
\end{equation}
where \(\alpha_K\) is the recovery rate on tangible capital and \(\alpha_S\) is the recovery rate on intangible capital, with \(\alpha_S < \alpha_K\) reflecting the lower pledgeability of intangibles \citep{holttinenmelolinnafroem2025}.

To ensure no default in equilibrium, banks lend only up to the recoverable collateral value at the risk-free rate. Given the timing structure described below, the collateral constraint is:
\begin{equation}
D_{j,t} \le \alpha_K K_{j,t} + \alpha_S S_{j,t}.
\label{eq:collateral_constraint}
\end{equation}

\begin{assumption}[Low Intangible Pledgeability]
\label{ass:pledgeability}
Intangible assets are less pledgeable than tangible assets: \(0 \le \alpha_S < \alpha_K < 1\). The baseline case considers \(\alpha_S=0\) (intangibles fully non-pledgeable).
\end{assumption}

\subsubsection{Within-Period Timing}

To clarify the sequence of decisions within each period \(t\), I specify the following timing. The stages below describe the logical sequence of decisions and cash flows within a single period; all labor $(L, H^P, H^R)$ is hired at the beginning of the period at the prevailing wages $(w_L, w_H)$, but the timing structure clarifies how different expenditures are financed:

\begin{enumerate}[leftmargin=*]
\item \textbf{Beginning of period:} Firm enters with state \((z_{t-1}, K_t, S_t, D_{t-1})\), where \(z_{t-1}\) denotes the lagged productivity realization and \(D_{t-1}\) is the debt obligation carried forward from the previous period.

\item \textbf{Productivity shock:} Firm draws current productivity \(z_t\) from the AR(1) process conditional on \(z_{t-1}\).

\item \textbf{Financial decisions:} Firm simultaneously repays old debt \(RD_{t-1}\) and obtains new debt \(D_t\) from banks, subject to the collateral constraint \eqref{eq:collateral_constraint} based on current capital stocks \((K_t, S_t)\). The net financial flow to the firm is \(D_t - RD_{t-1}\).

\item \textbf{Static production decisions:} Firm chooses unskilled labor \(L\) and skilled production labor \(H^P\) to maximize current gross profits. These are static choices determined by first-order conditions \eqref{eq:foc_L}, \eqref{eq:foc_HP}.

\item \textbf{Production:} Output \(Y\) is produced using current capital stocks \((K_t, S_t)\) and chosen labor \((L, H^P)\).

\item \textbf{Exit shock:} With probability \(\zeta\), firm exits the market. Exiting firms:
\begin{itemize}
\item Collect production revenue net of wages: \(Y - w_L L - w_H H^P\)
\item Liquidate undepreciated capital: \((1-\delta_K)K_t + (1-\delta_S)S_t\)
\item Repay current debt: \(D_t\)
\item Distribute residual to shareholders: \(\text{Div}^{exit} = Y - w_L L - w_H H^P + (1-\delta_K)K_t + (1-\delta_S)S_t - D_t\)
\end{itemize}
With probability \((1-\zeta)\), firm continues to Stage 7.

\item \textbf{Investment decisions (for survivors only):} Surviving firms simultaneously choose:
\begin{itemize}
\item Tangible investment \(I^K_{j,t}\), determining \(K_{j,t+1}\) via \eqref{eq:capital_accum_K}
\item R\&D labor \(H^R_{j,t}\), determining \(S_{j,t+1}\) via \eqref{eq:capital_accum_S}
\end{itemize}
These choices are subject to the budget constraint \eqref{eq:budget_constraint}, which requires that investment expenditures not exceed available resources from gross profits and net borrowing.

\item \textbf{End of period:} Surviving firms pay dividends (if any) and enter next period with state \((z_t, K_{j,t+1}, S_{j,t+1}, D_{j,t})\).
\end{enumerate}

\subsubsection{Budget Constraint}

Given the timing above, the firm's per-period budget constraint is:
\begin{equation}
I^K_{j,t} + w_L L_{j,t} + w_H H^P_{j,t} + w_H H^R_{j,t} + R D_{j,t-1} \le Y_{j,t} + D_{j,t}.
\label{eq:budget_constraint}
\end{equation}
\textbf{Timing and Wage Equality.} Although the within-period timing describes $H^P$ as hired for production (Step 4) and $H^R$ as hired for R\&D investment (Step 7), both types of skilled labor are employed within the same period $t$ and paid the same wage $w_H$. The timing structure describes the sequence of decisions and cash flows within the period, not separate labor markets. Skilled workers are hired at the beginning of period $t$ with the understanding that some will work in production while others work in R\&D. The single wage $w_H$ clears the aggregate skilled labor market: $\int (H^P_j + H^R_j) d\Psi = \bar{H}$.

We can rewrite this in terms of gross profits from production:
\[
\Pi^{gross}_{j,t} := Y_{j,t} - w_L L_{j,t} - w_H H^P_{j,t},
\]
so that:
\begin{equation}
I^K_{j,t} + w_H H^R_{j,t} + R D_{j,t-1} \le \Pi^{gross}_{j,t} + D_{j,t}.
\label{eq:budget_constraint_gross}
\end{equation}

Dividends (profits) distributed to shareholders are:
\begin{equation}
\text{Div}_{j,t} = Y_{j,t} - I^K_{j,t} - w_L L_{j,t} - w_H(H^P_{j,t} + H^R_{j,t}) - R D_{j,t-1} + D_{j,t}.
\label{eq:dividends}
\end{equation}

\subsection{Firm Problem}

\subsubsection{Recursive Formulation}

The firm's state is \((z, K, S, D_{-1})\). Given the timing structure, the firm's problem can be written recursively. Let \(\Pi^*(z, K, S)\) denote the maximized gross profits from production:
\begin{equation}
\Pi^*(z, K, S) = \max_{L,H^P} \Big\{ Y(z, K, S, L, H^P) - w_L L - w_H H^P \Big\}.
\label{eq:gross_profits}
\end{equation}

Then the value function satisfies:
\begin{align}
V(z, K, S, D_{-1}) &= \max_{D} \Bigg\{ \Pi^*(z, K, S) - R D_{-1} + D \nonumber \\
&\quad + \max_{I^K,H^R} \Big\{ - I^K - w_H H^R + \beta(1-\zeta) \E[V(z', K', S', D) \mid z] \Big\} \Bigg\},
\label{eq:value_function}
\end{align}
subject to:
\begin{itemize}
\item Capital accumulation: \(K' = (1-\delta_K)K + I^K\), \(S' = (1-\delta_S)S + \Gamma(H^R)^{\xi}\)
\item Collateral constraint: \(D \le \alpha_K K + \alpha_S S\)
\item Non-negative dividends: \(\text{Div} = \Pi^* - I^K - w_H H^R - R D_{-1} + D \ge 0\)
\end{itemize}

\textbf{Interpretation:} The firm first chooses debt \(D\) subject to the collateral constraint, then maximizes static production profits \(\Pi^*(z, K, S)\), and finally chooses investment \((I^K, H^R)\) subject to the non-negative dividend constraint to maximize continuation value. 

\subsubsection{First-Order Conditions}

Let \(\mu_{j,t}\) denote the shadow value of internal funds.\footnote{\textbf{Notational Convention:} Formally, $\mu_t \equiv 1 + \mu_t^{raw}$ where $\mu_t^{raw}$ is the Lagrange multiplier on the non-negative dividend constraint. This normalization ensures $\mu_t = 1$ when the constraint does not bind (dividends are positive), and $\mu_t > 1$ when internal funds are scarce (dividends are zero). This convention eliminates ``$1+$'' terms throughout the derivations.} Let \(\lambda_{j,t}\) denote the multiplier on the collateral constraint \eqref{eq:collateral_constraint}.

\paragraph{Static Production FOCs.} From \eqref{eq:gross_profits}:
\begin{equation}
\frac{\partial Y}{\partial L} = w_L,
\label{eq:foc_L}
\end{equation}
\begin{equation}
\frac{\partial Y}{\partial H^P} = w_H.
\label{eq:foc_HP}
\end{equation}
These determine \(L^*(z, K, S)\) and \(H^{P*}(z, K, S)\) as functions of the state.

\paragraph{Envelope Conditions.} Taking derivatives of \(V(z, K, S, D_{-1})\) with respect to state variables yields:
\begin{equation}
V_K(z, K, S, D_{-1}) = \mu \frac{\partial Y}{\partial K} + \lambda \alpha_K + (1-\delta_K)\beta(1-\zeta) \E[V_{K'}(z', K', S', D) \mid z],
\label{eq:envelope_K}
\end{equation}
\begin{equation}
V_S(z, K, S, D_{-1}) = \mu \frac{\partial Y}{\partial S} + \lambda \alpha_S + (1-\delta_S)\beta(1-\zeta) \E[V_{S'}(z', K', S', D) \mid z],
\label{eq:envelope_S}
\end{equation}
\begin{equation}
V_{D_{-1}}(z, K, S, D_{-1}) = -\mu R.
\label{eq:envelope_D}
\end{equation}

\textbf{Derivation note:} By the envelope theorem applied to \eqref{eq:gross_profits}, we have \(\partial \Pi^*/\partial K = \partial Y/\partial K\) (the static optimization does not involve \(\mu\)). In the dynamic value function \eqref{eq:value_function}, this cash flow is valued at the shadow price \(\mu\), yielding the first term in \eqref{eq:envelope_K}. The second term \(\lambda \alpha_K\) reflects the value of additional collateral, and the third term is the continuation value of undepreciated capital. Note that \(V_{D_{-1}}\) in equation \eqref{eq:envelope_D} denotes the derivative with respect to the predetermined debt stock \(D_{-1}\), as \(D_{-1}\) is a state variable while current debt \(D\) is a choice variable.

\paragraph{First-Order Condition for Tangible Investment.} From \(\partial \mathcal{L}/\partial I^K = 0\):
\begin{equation}
\mu_{j,t} = \beta(1-\zeta) \E[V_{K'}(z_{j,t+1}, K_{j,t+1}, S_{j,t+1}, D_{j,t}) \mid z_{j,t}].
\label{eq:foc_IK}
\end{equation}

Combining \eqref{eq:foc_IK} with the envelope condition \eqref{eq:envelope_K}:
\begin{equation}
\mu_{j,t} = \beta(1-\zeta) \E_{j,t}\Big[\mu_{j,t+1} \frac{\partial Y_{j,t+1}}{\partial K_{j,t+1}} + \lambda_{j,t+1} \alpha_K + (1-\delta_K)\mu_{j,t+1}\Big],
\label{eq:euler_K}
\end{equation}
where \(\E_{j,t}[\cdot] \equiv \E[\cdot \mid z_{j,t}]\) denotes the conditional expectation given the firm's current productivity.

\paragraph{First-Order Condition for Intangible Investment (via R\&D).} The firm chooses \(H^R\) to balance the wage cost against the marginal value of additional intangible capital. Since \(S' = (1-\delta_S)S + \Gamma(H^R)^{\xi}\), we have \(\partial S'/\partial H^R = \Gamma \xi (H^R)^{\xi-1}\).

The FOC with respect to \(H^R\) is:
\begin{equation}
\mu_{j,t} w_H = \beta(1-\zeta) \E[V_{S'}(z_{j,t+1}, K_{j,t+1}, S_{j,t+1}, D_{j,t}) \mid z_{j,t}] \cdot \Gamma \xi (H^R_{j,t})^{\xi-1}.
\label{eq:foc_HR}
\end{equation}

Using the envelope condition \eqref{eq:envelope_S} and substituting recursively, we obtain:\footnote{Undepreciated intangible capital must be valued at its marginal replacement cost $q_{S,t+1} \equiv w_{H,t+1}/[\Gamma\xi(H^R_{j,t+1})^{\xi-1}]$, not at unity. From the period $t+1$ FOC for $H^R_{t+1}$, we have $\beta(1-\zeta)\mathbb{E}_{t+1}[V_{S,t+2}] = \mu_{t+1}w_{H,t+1}/[\Gamma\xi(H^R_{j,t+1})^{\xi-1}]$. Substituting this expression into the envelope condition yields the correct continuation value term.}
\begin{equation}
\mu_{j,t} w_H = \beta(1-\zeta) \Gamma \xi (H^R_{j,t})^{\xi-1} \E_{j,t}\left[\mu_{j,t+1} \frac{\partial Y_{j,t+1}}{\partial S_{j,t+1}} + \lambda_{j,t+1} \alpha_S + (1-\delta_S)\mu_{j,t+1}\frac{w_{H,t+1}}{\Gamma\xi(H^R_{j,t+1})^{\xi-1}}\right].
\label{eq:euler_S}
\end{equation}

\paragraph{First-Order Condition for Debt.} The firm balances the benefit of additional borrowing today against the cost of repayment tomorrow. Given that the firm borrows \(D_t\) today and repays \(RD_t = D_t/\beta\) tomorrow (since \(R = 1/\beta\)), and accounting for the survival probability \((1-\zeta)\), the FOC is:
\begin{equation}
\mu_{j,t} = (1-\zeta) \E_{j,t}[\mu_{j,t+1}] + \lambda_{j,t}.
\label{eq:foc_debt}
\end{equation}


\subsubsection{Economic Interpretation}

The first-order conditions \eqref{eq:euler_K}, \eqref{eq:euler_S} reveal the core distortion created by financial frictions. Consider two firms with identical productivity \(z\) but different financial positions.

An unconstrained firm has \(\lambda = 0\), so its Euler equations reduce to standard intertemporal optimality conditions equating the marginal cost of investment today to the expected discounted marginal benefit tomorrow. The firm's investment decisions depend only on fundamentals: productivity, depreciation rates, and the interest rate.

A constrained firm has \(\lambda > 0\), introducing wedges into both Euler equations. From \eqref{eq:euler_K}, the presence of \(\lambda_{j,t+1} \alpha_K\) in the expectation makes tangible capital investment more attractive because tangible assets relax future borrowing constraints. Similarly, from \eqref{eq:euler_S}, the term \(\lambda_{j,t+1} \alpha_S\) provides an analogous benefit for intangible capital.

The crucial asymmetry arises from Assumption \ref{ass:pledgeability}: \(\alpha_K > \alpha_S\). When the collateral constraint binds, tangible capital provides greater future collateral value than intangible capital. This creates a pecking-order distortion where constrained firms tilt their investment composition toward tangibles, even when the fundamental marginal products would justify more intangible investment.

% \paragraph{Quantifying the Distortion.} To make this concrete, consider the relative magnitudes of collateral values. For a constrained firm with \(\E[\lambda_{t+1}] > 0\), the expected collateral benefit of tangible capital is \(\E[\lambda_{t+1}]\alpha_K\), while for intangible capital it is \(\E[\lambda_{t+1}]\alpha_S\). The pledgeability gap is:
% \begin{equation}
% \Delta \equiv \E[\lambda_{t+1}](\alpha_K - \alpha_S).
% \label{eq:distortion_gap}
% \end{equation}

% This quantity represents the differential incentive to invest in tangibles versus intangibles, beyond what fundamentals justify. The gap \(\Delta\) is:
% \begin{itemize}
% \item Increasing in the pledgeability differential \(\alpha_K - \alpha_S\)
% \item Increasing in the expected tightness of constraints \(\E[\lambda_{t+1}]\)
% \item Zero for unconstrained firms (\(\lambda = 0\))
% \end{itemize}

% In the baseline case where \(\alpha_S = 0\) (intangibles are fully non-pledgeable), this simplifies to \(\Delta = \E[\lambda_{t+1}]\alpha_K\), and the entire collateral value of capital goes to tangibles. This distortion is formalized and characterized in subsection VII below.

\subsection{Equilibrium}

\subsubsection{Definition}

\begin{definition}[Stationary Recursive Equilibrium]
\label{def:equilibrium}
A stationary recursive equilibrium consists of:
\begin{enumerate}[label=(\roman*)]
\item Firm value functions \(V(z, K, S, D_{-1})\) and policy functions \(\{K'(z, K, S, D_{-1}), \\ S'(\cdot), D(\cdot), L(\cdot), H^P(\cdot), H^R(\cdot), I^K(\cdot)\}\),
\item Wages \((w_L, w_H)\) and interest rate \(R\),
\item A stationary distribution of firms \(\Psi^*(z, K, S, D_{-1})\),
\item Aggregate quantities \(\{C, K_{agg}, S_{agg}, L_{agg}, H_{agg}, D_{agg}, Y_{agg}\}\),
\end{enumerate}
such that:
\begin{enumerate}[label=(\alph*)]
\item Firms optimize: \(V(z, K, S, D_{-1})\) solves \eqref{eq:value_function} and policies satisfy FOCs.
\item Household optimizes: \(R = 1/\beta\) from Euler equation \eqref{eq:interest_rate}.
\item Labor markets clear:
\[
\int L_j d\Psi^* = \bar L, \qquad \int (H^P_j + H^R_j) d\Psi^* = \bar H.
\]
\item Credit market clears:
\[
D_{agg} := \int D_j d\Psi^* = D^H.
\]
\item Goods market clears:
\[
C + \int I^K_j d\Psi^* + \zeta c_e N = Y_{agg} := \int Y_j d\Psi^*,
\]
where \(N\) is the measure of firms (\(N = \int d\Psi^* = 1\) by normalization).
\item Free entry: Expected value of entrants equals entry cost:
\[
\int V(z, 0, 0, 0) dF_z(z) = c_e.
\]
\item Stationarity: \(\Psi^*\) is invariant under the transition implied by policy functions, exit, and entry.
\end{enumerate}
\end{definition}

\section{Quantitative Experiments}
\label{sec:plan}

The quantitative model enables several counterfactual experiments to quantify the skill-biased stagnation mechanism.

\paragraph{Baseline Experiment: Skill Supply Shock.} I compute steady-state economies for different values of the skill ratio $\bar{H}/\bar{L}$, comparing outcomes with financial frictions (baseline calibration) against a frictionless counterfactual (setting $\lambda=0$ for all firms). This experiment measures how financial frictions mediate the aggregate productivity response to skill accumulation.

\paragraph{Transition Dynamics.} An alternative approach computes the full transition path following an unexpected permanent increase in $\bar{H}/\bar{L}$, tracking how constrained and unconstrained firms adjust their capital stocks and employment over time. This captures the dynamic misallocation channel as financially constrained firms slowly build intangible capital.

\paragraph{Financial Liberalization.} A third experiment varies the pledgeability of intangibles $\alpha_S$ while holding technology fixed, simulating financial innovations that improve the collateralizability of intangible assets. The baseline calibration uses $\alpha_S = 0.134$ and $\alpha_K = 0.381$ from \citet{holttinen2025aggregate}. Increasing $\alpha_S$ toward $\alpha_K$ quantifies the TFP gains from reducing the pledgeability gap. 

\clearpage
\appendix

\section{Data Construction and Cleaning}
\label{app:data_cleaning}

\subsection{Data Sources and Sample Construction}

The empirical analysis uses Portuguese firm-level data from two administrative sources for the period 2011-2022:

\begin{enumerate}[label=(\roman*)]
\item \textbf{SCIE (Sistema de Contas Integradas das Empresas):} Balance sheet and income statement data covering all Portuguese firms required to file annual accounts. Variables include assets, liabilities, equity, revenue, costs, investment flows, and R\&D expenditures.

\item \textbf{QP (Quadros de Pessoal):} Matched employer-employee data containing firm characteristics and worker-level information. I aggregate worker data to construct firm-level skill composition measures, defining skilled workers as those with tertiary education (ISCED 5-8).
\end{enumerate}

Before the cleaning procedure, I restrict the sample to private incorporated businesses (\textit{sociedades}), as balance sheet data is only available for incorporated firms and public firms face different objectives and constraints. The initial merged dataset (SCIE-matched, private incorporated firms) contains 2,329,807 firm-year observations spanning 2011-2022.

\subsection{Price Deflators}

All monetary variables are deflated to constant 2020 prices using three price indices:

\begin{itemize}
\item \textbf{GDP deflator} (base year 2020=100): Applied to revenue, costs, wages, and R\&D expenditures. Source: FRED (Federal Reserve Economic Data), series PRTGDPDEFQISMEI\_NBD20200101.

\item \textbf{GFCF deflator} (Gross Fixed Capital Formation, 2020=100): Applied to investment and disinvestment flows. Source: EU KLEMS-INTAN database, Portuguese data.

\item \textbf{Capital deflator} (2020=100): Applied to balance sheet stocks. Source: EU KLEMS-INTAN database, Portuguese data.
\end{itemize}

For 2022, GFCF and capital deflators are extrapolated using the 2021-2022 growth rate of the GDP deflator, as EU KLEMS-INTAN data availability ends in 2021.

\subsection{Data Cleaning Procedure}

The cleaning procedure follows a sequential pipeline to ensure data quality before constructing intangible capital stocks. All monetary variables are first deflated to constant 2020 prices using appropriate deflators: GDP deflator for revenue, costs, and wages; GFCF deflator for investment flows; and capital deflator for balance sheet stocks.

\paragraph{Step 1: Structural Problems.} Remove observations with missing firm identifiers or duplicate firm-year observations. This quality control check drops a negligible number of observations.

\paragraph{Step 2: Missing or Negative Values.} Drop observations with missing or negative values in key balance sheet and income statement variables. Since deflated monetary values should be non-negative, negative values indicate data errors. Checked variables include:
\begin{itemize}
\item Tangible fixed assets (physical capital)
\item Balance sheet intangibles (excluding goodwill)
\item Revenue, production value, gross value added, wagebill
\item Total assets, long-term debt, short-term debt, interest expenses
\item Tangible investment flows
\item Intangible construction inputs: R\&D expenditures, advertising, training
\end{itemize}

\paragraph{Step 3: Missing Economic Activity.} Drop observations indicating absence of genuine economic activity. Specifically, I remove firms with:
\begin{itemize}
\item Zero or missing number of workers
\item Less than 1,000 euros in tangible fixed assets
\item Less than 1,000 euros in revenue
\item Less than 1,000 euros in production value
\item Less than 500 euros in wagebill
\end{itemize}
These thresholds eliminate shell companies and data errors while retaining small active firms. Physical capital is defined as tangible fixed assets following standard practice in production function estimation, excluding inventories and working capital.

\paragraph{Step 4: Panel Structure.} Require at least two consecutive observations per firm to enable construction of intangible capital stocks via the perpetual inventory method (PIM). Firms with isolated observations cannot be used for dynamic capital stock accumulation.

\paragraph{Final Sample.} The sequential cleaning procedure yields a final analytical sample of \textbf{1,759,093 firm-year observations} spanning 2011-2022. All observations have complete data for core balance sheet variables, positive economic activity, and sufficient panel length for capital stock construction. Missing values in R\&D, advertising, or training expenditures are treated as zero investment in these categories, following standard practice in the intangible capital literature. 

\subsection{Intangible Capital Construction}

Following \citet{peterstaylor2017}, I construct intangible capital stocks using the perpetual inventory method (PIM). The baseline measure combines two components: knowledge capital from R\&D and balance sheet intangibles.

\paragraph{Knowledge Capital (from R\&D).} Accumulated from reported R\&D expenditures using sector-specific depreciation rates from \citet{ewenspeterswang2025}:
\begin{equation}
K^{knowledge}_{j,t} = (1-\delta^{sector}_{R}) K^{knowledge}_{j,t-1} + RD_{j,t},
\end{equation}
where $\delta^{sector}_{R}$ varies by Fama-French 5 industry classification:
\begin{itemize}
\item Consumer: $\delta_R = 0.43$
\item Manufacturing: $\delta_R = 0.50$
\item High Tech: $\delta_R = 0.42$
\item Health: $\delta_R = 0.33$
\item Other: $\delta_R = 0.35$
\end{itemize}
Initial stocks are set to zero ($K^{knowledge}_{j,2011} = 0$). The sector-specific rates capture heterogeneity in knowledge obsolescence across industries: manufacturing R\&D depreciates faster (50\%) than health-related R\&D (33\%), reflecting differences in product cycles and technological change.

\paragraph{Balance Sheet Intangibles.} Externally acquired intangibles recorded on firm balance sheets, measured directly as stock variables (excluding goodwill). These capture purchased patents, software, databases, and other codified intangible assets.

\paragraph{Total Intangible Capital.} The baseline measure sums knowledge capital and balance sheet intangibles:
\begin{equation}
K^{intangible}_{j,t} = K^{knowledge}_{j,t} + K^{external}_{j,t}.
\end{equation}
This \emph{BS+R\&D} measure focuses on intangibles with clearer market values and avoids the measurement challenges associated with organization capital (SG\&A-based accumulation). Total capital is $K^{total}_{j,t} = K^{physical}_{j,t} + K^{intangible}_{j,t}$, where physical capital equals tangible fixed assets from balance sheets.

\subsection{Winsorization}

To limit the influence of extreme outliers while preserving genuine economic variation, I apply winsorization to key variables by year and sector (3-digit CAE code). Financial variables and investment rates are winsorized at the 5th and 95th percentiles within each year-sector cell, allowing for industry heterogeneity in distributions while mitigating the impact of extreme values. This approach balances outlier treatment with preserving cross-sectional variation that is economically meaningful.

\subsection{Final Dataset Structure}

The final analysis-ready dataset contains 1,759,093 firm-year observations with:
\begin{itemize}
\item All monetary variables in constant 2020 prices
\item Constructed intangible capital stocks using sector-specific depreciation
\item Firm-level skill composition (share of workers with tertiary education, share of R\&D workers)
\item Multiple intangible capital definitions (BS+R\&D baseline, full Peters-Taylor alternative)
\item Investment rates and intensity measures (intangible intensity, R\&D intensity)
\item Financial constraint proxies (leverage, interest rates, credit spreads)
\item Winsorized versions of key variables
\end{itemize}


\section{Robustness: Alternative Outcome Measures}
\label{app:robustness}

This appendix presents robustness checks for the main empirical results using alternative outcome measures. The main text focuses on gross value added (GVA) as the preferred output measure because it captures the firm's contribution to aggregate output net of intermediate inputs. Here I show that the key findings are robust to using revenue or production value as alternative measures.

\subsection{Fact 1: Complementarity}

Table \ref{tab:complementarityrobust} replicates the complementarity analysis from Table \ref{tab:complementarity} using log revenue (columns 1--3) and log production value (columns 4--6) as dependent variables. Each panel follows the same nested specification: no controls, fixed effects only, and fixed effects with controls.

\begin{table}[htbp]\centering\small
\def\sym#1{\ifmmode^{#1}\else\(^{#1}\)\fi}
\caption{Complementarity Between Intangibles and Skilled Labor: Alternative Outcomes\label{tab:complementarityrobust}}
\begin{tabular}{l*{6}{c}}
\toprule
            &\multicolumn{3}{c}{Log Revenue}                                  &\multicolumn{3}{c}{Log Production}                               \\\cmidrule(lr){2-4}\cmidrule(lr){5-7}
            &\multicolumn{1}{c}{(1)}&\multicolumn{1}{c}{(2)}&\multicolumn{1}{c}{(3)}&\multicolumn{1}{c}{(4)}&\multicolumn{1}{c}{(5)}&\multicolumn{1}{c}{(6)}\\
\midrule
Intangible Intensity&       0.64\sym{***}&      -0.05\sym{***}&      -0.10\sym{***}&       0.48\sym{***}&      -0.05\sym{***}&      -0.11\sym{***}\\
            &     (0.01)         &     (0.01)         &     (0.01)         &     (0.01)         &     (0.01)         &     (0.01)         \\
\addlinespace
Share Skilled Workers&       0.18\sym{***}&      -0.05\sym{***}&      -0.01\sym{***}&       0.36\sym{***}&      -0.05\sym{***}&      -0.01\sym{**} \\
            &     (0.00)         &     (0.00)         &     (0.00)         &     (0.00)         &     (0.00)         &     (0.00)         \\
\addlinespace
Intangible Intensity $\times$ Share Skilled&       0.98\sym{***}&       0.23\sym{***}&       0.03\sym{*}  &       1.18\sym{***}&       0.25\sym{***}&       0.04\sym{***}\\
            &     (0.03)         &     (0.02)         &     (0.01)         &     (0.03)         &     (0.02)         &     (0.01)         \\
\midrule
Observations&   1,759,093         &   1,759,093         &   1,759,084         &   1,759,093         &   1,759,093         &   1,759,084         \\
Adjusted R-squared&       0.012         &       0.933         &       0.952         &       0.017         &       0.931         &       0.951         \\
Firm FE     &          No         &         Yes         &         Yes         &          No         &         Yes         &         Yes         \\
Year FE     &          No         &         Yes         &         Yes         &          No         &         Yes         &         Yes         \\
Industry FE &          No         &         Yes         &         Yes         &          No         &         Yes         &         Yes         \\
Controls    &          No         &          No         &         Yes         &          No         &          No         &         Yes         \\
\bottomrule
\multicolumn{7}{l}{\footnotesize Intangible intensity = $K_{intangible}$ / $K_{total}$.}\\
\multicolumn{7}{l}{\footnotesize Controls include log total capital, log employment, and log firm age.}\\
\multicolumn{7}{l}{\footnotesize Robust standard errors in parentheses.}\\
\end{tabular}
\end{table}


The interaction between intangible intensity and skilled labor share is positive and statistically significant across all specifications and outcome measures. The magnitude of the complementarity effect is similar across GVA, revenue, and production, confirming that the technological complementarity documented in the main text is not driven by the choice of output measure.

\subsection{Fact 3: Underexploitation by Leverage}

Table \ref{tab:underexploitationrobust} replicates the underexploitation analysis from Table \ref{tab:underexploitation} using log revenue (columns 1--3) and log production value (columns 4--6) as dependent variables. Each panel reports results for all firms, low-leverage firms, and high-leverage firms.

\begin{table}[htbp]\centering\small
\def\sym#1{\ifmmode^{#1}\else\(^{#1}\)\fi}
\caption{Underexploitation of Complementarity: Alternative Outcomes\label{tab:underexploitationrobust}}
\begin{tabular}{l*{6}{c}}
\toprule
            &\multicolumn{3}{c}{Log Revenue}                                  &\multicolumn{3}{c}{Log Production}                               \\\cmidrule(lr){2-4}\cmidrule(lr){5-7}
            &\multicolumn{1}{c}{All}&\multicolumn{1}{c}{Low Lev.}&\multicolumn{1}{c}{High Lev.}&\multicolumn{1}{c}{All}&\multicolumn{1}{c}{Low Lev.}&\multicolumn{1}{c}{High Lev.}\\
\midrule
\textit{Main effects:}&                     &                     &                     &                     &                     &                     \\
\addlinespace
Intangible Intensity&      -0.10\sym{***}&      -0.11\sym{***}&      -0.10\sym{***}&      -0.11\sym{***}&      -0.12\sym{***}&      -0.11\sym{***}\\
            &     (0.01)         &     (0.01)         &     (0.01)         &     (0.01)         &     (0.01)         &     (0.01)         \\
\addlinespace
Share Skilled Workers&      -0.01\sym{***}&      -0.02\sym{***}&      -0.01         &      -0.01\sym{**} &      -0.02\sym{***}&      -0.00         \\
            &     (0.00)         &     (0.01)         &     (0.01)         &     (0.00)         &     (0.01)         &     (0.00)         \\
\addlinespace
\textit{Complementarity:}&                     &                     &                     &                     &                     &                     \\
\addlinespace
Intangible Intensity $\times$ Share Skilled&       0.03\sym{*}  &       0.04\sym{**} &      -0.00         &       0.04\sym{***}&       0.07\sym{***}&       0.00         \\
            &     (0.01)         &     (0.02)         &     (0.02)         &     (0.01)         &     (0.02)         &     (0.02)         \\
\midrule
Observations&   1,759,084         &     852,528         &     856,243         &   1,759,084         &     852,528         &     856,243         \\
Adjusted R-squared&       0.952         &       0.957         &       0.958         &       0.951         &       0.955         &       0.959         \\
\bottomrule
\multicolumn{7}{l}{\footnotesize Low- and high-leverage defined relative to sector-year median leverage.}\\
\multicolumn{7}{l}{\footnotesize All specifications include firm, year, and industry fixed effects.}\\
\multicolumn{7}{l}{\footnotesize Controls include log total capital, log employment, and log firm age.}\\
\multicolumn{7}{l}{\footnotesize Robust standard errors in parentheses.}\\
\end{tabular}
\end{table}


The pattern of underexploitation is robust across outcome measures: the complementarity coefficient is positive and significant for low-leverage firms but small and insignificant for high-leverage firms. This confirms that the differential ability to exploit intangible-skill complementarity by financial constraint status is not an artifact of the GVA measure.


\bibliographystyle{apalike}
\bibliography{references}

\end{document}