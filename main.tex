\documentclass[12pt]{article}
\usepackage[a4paper, margin=1in]{geometry}
\usepackage{amsmath, amsfonts, amssymb, amsthm, bm}
\usepackage{graphicx, caption, subcaption, float, booktabs}
\usepackage{enumitem, hyperref, natbib, xcolor, dsfont, scrextend, authblk}
\usepackage[bottom]{footmisc}

\deffootnote{1.5em}{0em}{\thefootnotemark\quad}
\renewcommand\thesection{\Roman{section}.}
\renewcommand\thesubsection{\thesection\Alph{subsection}.}
\renewcommand\thesubsubsection{\arabic{subsubsection}.}

\hypersetup{
  colorlinks=true,
  linkcolor=blue,
  citecolor=blue,
  urlcolor=cyan
}

\newtheorem{proposition}{Proposition}
\newtheorem{hypothesis}{Hypothesis}
\newtheorem{lemma}{Lemma}
\newtheorem{theorem}{Theorem}
\newtheorem{definition}{Definition}
\newtheorem{corollary}{Corollary}
\newtheorem{assumption}{Assumption}
\newtheorem*{remark}{Remark}

\DeclareMathOperator*{\argmax}{arg\,max}
\DeclareMathOperator*{\E}{\mathbb{E}}

% Figure note command
\newcommand{\note}[1]{\par\vspace{6pt}\footnotesize\textit{Note:} #1}

\title{\textbf{Skill-biased stagnation}}
\author{Alejandro Vicente\thanks{University of Alicante. Email: alejandro.vicente@ua.es}}
\date{\today \\ \vspace{0.25cm} \textcolor{red}{PRELIMINARY AND INCOMPLETE, DO NOT CIRCULATE}}

\begin{document}
\maketitle

\begin{abstract}
This paper studies how financial frictions prevent economies from fully exploiting productivity gains from rising human capital. Using Portuguese matched employer-employee and firm balance sheet data for 2011-2022, I document three empirical facts: (1) intangible capital and skilled labor are complements in production, (2) financially constrained firms underinvest in intangibles and R\&D labor, and (3) constrained firms underexploit the complementarity: during Portugal's skill supply expansion, wage premia declined more sharply at constrained firms despite hiring skilled workers at similar rates. I develop a quantitative heterogeneous-firm model with nested CES production, endogenous intangible accumulation through R\&D, and collateral constraints with differential pledgeability (\(\alpha_K > \alpha_S\)). The model rationalizes the empirical patterns and reveals a \emph{skill-biased stagnation} mechanism: when skilled labor becomes more abundant, unconstrained firms exploit the complementarity by investing in both intangibles and skills, while constrained firms hire skilled workers but cannot finance complementary intangible investments, leading to underexploitation and muted aggregate productivity gains. Counterfactual experiments quantify the role of financial frictions in limiting returns to human capital accumulation.
\end{abstract}

\bigskip
\noindent \textbf{Keywords:} Intangible capital, financial frictions, skilled labor, productivity, firm dynamics

\noindent \textbf{JEL Classification:} E22, E24, G32

\newpage

\section{Introduction}

Over the past decades, advanced economies have experienced substantial increases in educational attainment alongside rising importance of intangible capital in production. Yet aggregate productivity growth has been disappointing. This paper proposes a mechanism through which financial frictions can prevent economies from fully exploiting the potential gains from higher human capital: \emph{skill-biased stagnation}.

The mechanism operates through three key channels. First, intangible capital and skilled labor are complements in production, implying that firms with high intangible intensity benefit more from skilled workers. Second, intangibles are harder to pledge as collateral than tangible assets, creating a pecking-order distortion where financially constrained firms underinvest in intangibles. Third, when skilled labor becomes more abundant, unconstrained firms exploit the complementarity by investing in both intangibles and skilled workers, while constrained firms cannot finance the necessary intangible investments and therefore underexploit their skilled labor.

I develop this argument in two stages. Using Portuguese matched employer-employee and balance sheet data for 2011-2022, I establish three empirical facts: (1) intangible intensity and skilled labor share are complements in production, (2) financially constrained firms invest less in intangibles and R\&D labor, and (3) the complementarity between intangibles and skills is significantly weaker for constrained firms, with dynamic evidence showing that wage premia declined more sharply at constrained firms during the skill supply expansion. I then build a quantitative heterogeneous-firm model with nested CES production technology, endogenous intangible capital accumulation through R\&D, and collateral constraints with differential pledgeability. The model rationalizes the empirical patterns and enables counterfactual analysis of skill supply shocks under different financial friction scenarios.

The paper contributes to three literatures. First, it connects to work on intangible capital and firm dynamics \citep{peterstaylor2017,atkeson2010innovation}, emphasizing the role of complementarities with skilled labor. Second, it relates to the literature on financial frictions and capital misallocation \citep{buera2013finance,moll2014aggregate}, highlighting the specific distortions created by differential pledgeability of capital types. Third, it speaks to the productivity slowdown debate by proposing financial frictions as a mechanism limiting the returns to human capital accumulation.

The remainder of the paper is organized as follows. Section \ref{sec:data_evidence} presents the data and empirical evidence. Section \ref{sec:toy_model} develops intuition with a stylized two-period model. Section \ref{sec:quantitative_model} presents the full quantitative model. Section \ref{sec:plan} outlines the quantitative experiments. The Appendix contains details on data construction and cleaning procedures.

\section{Data and Empirical Evidence}
\label{sec:data_evidence}

\subsection{Data Sources and Measurement}

The empirical analysis uses Portuguese firm-level data from two administrative sources for 2011-2022. The \textbf{SCIE (Sistema de Contas Integradas das Empresas)} provides balance sheet and income statement data for all Portuguese firms required to file annual accounts, including assets, liabilities, revenue, costs, and R\&D expenditures. The \textbf{QP (Quadros de Pessoal)} is a matched employer-employee dataset containing worker-level information that I aggregate to construct firm-level skill composition measures.

I define skilled workers as those with tertiary education (ISCED 5-8) and construct the share of skilled workers as the key skill measure. Intangible capital stocks are built using the perpetual inventory method following a simplified version of the approach by \citet{peterstaylor2017}, combining knowledge capital from R\&D expenditures and balance sheet intangibles. All monetary variables are deflated to constant 2020 prices. After cleaning and sample restrictions (positive employment, non-negative balance sheet items, at least two consecutive observations per firm), the final analytical sample contains 1,759,093 firm-year observations. Details on data construction and cleaning are provided in Appendix \ref{app:data_cleaning}.

Financial constraints are proxied by leverage (total debt relative to total assets). Following standard practice in the corporate finance literature, I classify firms as constrained if their leverage exceeds the sector-year median, allowing for industry heterogeneity in optimal capital structure.

\subsection{Empirical Facts}

\subsubsection*{Fact 1: Complementarity Between Intangibles and Skilled Labor}

To test for complementarity between intangibles and skilled labor, I estimate the following production function specification:
\begin{equation}
\begin{aligned}
\ln Y_{jt} =\;& \beta_1 \, \text{IntangIntensity}_{jt}
+ \beta_2 \, \text{ShareSkilled}_{jt}
+ \beta_3 \, (\text{IntangIntensity} \times \text{ShareSkilled})_{jt} \\
& + \bm{X}'_{jt}\bm{\gamma}
+ \alpha_j + \delta_t + \eta_i + \varepsilon_{jt},
\end{aligned}
\label{eq:complementarity}
\end{equation}

where $Y_{jt}$ is gross value added for firm $j$ in year $t$, $\text{IntangIntensity}_{jt} \equiv K^{intang}_{jt}/K^{total}_{jt}$ is the ratio of intangible to total capital, $\text{ShareSkilled}_{jt}$ is the share of workers with tertiary education, $\bm{X}_{jt}$ includes log total capital, log employment, and log firm age, and $\alpha_j$, $\delta_t$, $\eta_i$ denote firm, year, and industry fixed effects respectively. The coefficient of interest is $\beta_3$: a positive estimate indicates that the marginal product of skilled labor is increasing in intangible intensity, i.e., complementarity.

Table \ref{tab:complementarity} reports estimates of equation \eqref{eq:complementarity}. The table presents a nested specification: column (1) includes no controls or fixed effects, column (2) adds firm, year, and industry fixed effects, and column (3) further includes the control variables $\bm{X}_{jt}$.

\begin{table}[htbp]\centering
\def\sym#1{\ifmmode^{#1}\else\(^{#1}\)\fi}
\caption{Complementarity Between Intangibles and Skilled Labor\label{tab:complementarity}}
\begin{tabular}{l*{3}{c}}
\toprule
            &\multicolumn{1}{c}{(1)}&\multicolumn{1}{c}{(2)}&\multicolumn{1}{c}{(3)}\\
            &\multicolumn{1}{c}{Revenue}&\multicolumn{1}{c}{Production}&\multicolumn{1}{c}{GVA}\\
\midrule
Intangible Intensity&     -0.1018\sym{***}&     -0.1111\sym{***}&     -0.1254\sym{***}\\
            &    (0.0065)         &    (0.0061)         &    (0.0089)         \\
\addlinespace
Share Skilled Workers&     -0.0128\sym{***}&     -0.0084\sym{**} &     -0.0154\sym{***}\\
            &    (0.0036)         &    (0.0033)         &    (0.0046)         \\
\addlinespace
Intangible Intensity $\times$ Share Skilled&      0.0262\sym{*}  &      0.0417\sym{***}&      0.0398\sym{**} \\
            &    (0.0137)         &    (0.0125)         &    (0.0175)         \\
\midrule
Observations&   1,759,084         &   1,759,084         &   1,759,076         \\
Adjusted R-squared&       0.952         &       0.951         &       0.895         \\
\bottomrule
\multicolumn{4}{l}{\footnotesize All output measures in logs. Intangible intensity = K{intangible}$ / K{total}$.}\\
\multicolumn{4}{l}{\footnotesize All specifications include firm, year, and industry fixed effects.}\\
\multicolumn{4}{l}{\footnotesize Robust standard errors in parentheses.}\\
\end{tabular}
\end{table}


The interaction coefficient is positive and statistically significant across all specifications, demonstrating robust complementarity between intangibles and skilled labor. The coefficient attenuates from 0.111 without controls to 0.040 with the full set of fixed effects and controls, indicating that the raw correlation partly reflects selection (intangible-intensive firms hiring more skilled workers) but a substantial within-firm complementarity remains. This pattern indicates that the marginal product of skilled labor is increasing in intangible intensity: firms with higher intangible capital benefit more from employing skilled workers. Conversely, the marginal product of intangibles is increasing in skill share. This complementarity is precisely the technological structure embedded in the nested CES production function of the quantitative model. Table \ref{tab:complementarityrobust} in Appendix \ref{app:robustness} shows that results are robust to using revenue or production value as alternative outcome measures.

Figure \ref{fig:fact1_correlation} shows the raw correlation between intangible intensity and skill share using a binscatter with 20 quantiles. The positive relationship is evident even without controlling for firm characteristics, though the regression estimates exploit within-firm variation and control for confounding factors.

\begin{figure}[htbp]
\centering
\includegraphics[width=0.7\textwidth]{QP_SCIE/results/fact1_raw_correlation.png}
\caption{Raw Correlation Between Intangible Intensity and Skilled Labor Share}
\label{fig:fact1_correlation}
\end{figure}

\subsubsection*{Fact 2: Financial Constraints and the Pecking Order}

The model predicts that financially constrained firms underinvest in intangibles relative to tangibles because intangibles have lower pledgeability. Figure \ref{fig:fact2_pecking_order} presents residualized binscatters showing the relationship between leverage (the constraint proxy) and various investment decisions, after partialling out firm, year, and industry fixed effects along with log capital, employment, and age.

\begin{figure}[htbp]
\centering
\begin{subfigure}[b]{0.48\textwidth}
\includegraphics[width=\textwidth]{QP_SCIE/results/fact2_physical_inv_leverage.png}
\caption{Physical Investment Rate}
\end{subfigure}
\hfill
\begin{subfigure}[b]{0.48\textwidth}
\includegraphics[width=\textwidth]{QP_SCIE/results/fact2_intangible_inv_leverage.png}
\caption{Intangible Investment Rate}
\end{subfigure}

\vspace{0.5cm}

\begin{subfigure}[b]{0.48\textwidth}
\includegraphics[width=\textwidth]{QP_SCIE/results/fact2_intang_intensity_leverage.png}
\caption{Intangible Intensity}
\end{subfigure}
\hfill
\begin{subfigure}[b]{0.48\textwidth}
\includegraphics[width=\textwidth]{QP_SCIE/results/fact2_rd_workers_leverage.png}
\caption{Share of R\&D Workers}
\end{subfigure}

\caption{Financial Constraints and the Pecking Order Distortion}
\label{fig:fact2_pecking_order}
\note{All variables residualized by regressing on firm, year, and industry fixed effects, log total capital, log employment, and log firm age. Binscatters use 20 quantiles. Panel (d) restricts to firms with positive R\&D employment.}
\end{figure}

The evidence reveals a clear pecking-order pattern. While physical investment shows little systematic relationship with leverage (Panel a), intangible investment, intangible intensity, and R\&D worker allocation all decline sharply with leverage (Panels b-d). Constrained firms shift their investment composition toward tangible assets and allocate fewer skilled workers to R\&D activities, precisely as the model predicts when intangibles have lower collateral value than tangibles.

\subsubsection*{Fact 3: Underexploitation of Complementarity by Constrained Firms}

If financial constraints prevent firms from building intangible capital, they should also reduce the extent to which firms can exploit the technological complementarity between intangibles and skills. Table \ref{tab:underexploitation} tests this prediction by estimating equation \eqref{eq:complementarity} separately for the pooled sample (column 1), low-leverage firms (column 2), and high-leverage firms (column 3), where leverage groups are defined relative to the sector-year median.

\begin{table}[htbp]\centering
\def\sym#1{\ifmmode^{#1}\else\(^{#1}\)\fi}
\caption{Underexploitation of Intangibles-Skills Complementarity\label{tab:underexploitation}}
\begin{tabular}{l*{3}{c}}
\toprule
            &\multicolumn{3}{c}{Log Gross Value Added}                        \\\cmidrule(lr){2-4}
            &\multicolumn{1}{c}{All Firms}&\multicolumn{1}{c}{Low Leverage}&\multicolumn{1}{c}{High Leverage}\\
\midrule
\textit{Main effects:}&                     &                     &                     \\
\addlinespace
Intangible Intensity&      -0.13\sym{***}&      -0.13\sym{***}&      -0.12\sym{***}\\
            &     (0.01)         &     (0.01)         &     (0.01)         \\
\addlinespace
Share Skilled Workers&      -0.02\sym{***}&      -0.02\sym{***}&      -0.01         \\
            &     (0.00)         &     (0.01)         &     (0.01)         \\
\addlinespace
\textit{Complementarity:}&                     &                     &                     \\
\addlinespace
Intangible Intensity $\times$ Share Skilled&       0.04\sym{**} &       0.06\sym{**} &      -0.01         \\
            &     (0.02)         &     (0.03)         &     (0.03)         \\
\midrule
Observations&   1,759,076         &     852,524         &     856,239         \\
Adjusted R-squared&       0.895         &       0.904         &       0.906         \\
\bottomrule
\multicolumn{4}{l}{\footnotesize Dependent variable: Log gross value added (GVA).}\\
\multicolumn{4}{l}{\footnotesize Low- and high-leverage defined relative to sector-year median leverage.}\\
\multicolumn{4}{l}{\footnotesize All specifications include firm, year, and industry fixed effects.}\\
\multicolumn{4}{l}{\footnotesize Controls include log total capital, log employment, and log firm age.}\\
\multicolumn{4}{l}{\footnotesize Robust standard errors in parentheses.}\\
\end{tabular}
\end{table}


The interaction coefficient is 0.055 and significant for low-leverage firms (column 2), but statistically indistinguishable from zero ($-0.005$, p $>$ 0.10) for high-leverage firms (column 3). This stark difference indicates that constrained firms cannot exploit the complementarity: even when they employ skilled workers, the lack of complementary intangible capital limits productivity gains. The pooled specification (column 1) masks this heterogeneity, showing an intermediate coefficient of 0.040. Table \ref{tab:underexploitationrobust} in Appendix \ref{app:robustness} confirms that this pattern holds when using revenue or production value as alternative outcome measures.

\paragraph{Dynamic Evidence: Wage Premium Decline.} The cross-sectional evidence establishes underexploitation at a point in time. I now examine the dynamic implications over the 2011-2022 period, during which Portugal experienced a substantial skill supply expansion. Figure \ref{fig:fact3_aggregate_trends} plots the evolution of the skill share and wage premium. The skill share increased substantially, yet the wage premium declined, a puzzle for standard capital-skill complementarity models where skill supply increases induce capital deepening that sustains premia.

\begin{figure}[htbp]
\centering
\includegraphics[width=0.72\textwidth]{QP_SCIE/results/fact3_skill_premium_trends.png}
\caption{Skill Supply Expansion and Declining Wage Premium}
\label{fig:fact3_aggregate_trends}
\note{Averages across all firms. Skill share is the fraction of workers with tertiary education. Wage premium is the ratio of average skilled to unskilled wages within firms.}
\end{figure}

Figure \ref{fig:fact3_leverage_trends} decomposes this pattern by financial constraints. Panel (a) shows that both low- and high-leverage firms increased their skilled labor shares over the period at remarkably similar rates. However, Panel (b) reveals that the wage premium decline was much sharper at high-leverage (constrained) firms. This differential trend reveals the mechanism: constrained firms hire skilled workers at similar rates to unconstrained firms, but cannot finance the complementary intangible investments needed to maintain skilled workers' productivity, causing the wage premium to compress more sharply.

\begin{figure}[htbp]
\centering
\begin{subfigure}[b]{0.48\textwidth}
\includegraphics[width=\textwidth]{QP_SCIE/results/fact3_skills_by_leverage.png}
\caption{Skill Share by Leverage}
\end{subfigure}
\hfill
\begin{subfigure}[b]{0.48\textwidth}
\includegraphics[width=\textwidth]{QP_SCIE/results/fact3_premium_by_leverage.png}
\caption{Wage Premium by Leverage}
\end{subfigure}
\caption{Skill Adoption vs.\ Wage Premium by Financial Constraints}
\label{fig:fact3_leverage_trends}
\note{Averages by leverage group. Low/high leverage defined relative to sector-year median. Both panels show residualized values, controlling for firm size (log capital, log employment), age, and industry fixed effects. Constrained firms hire skilled workers at similar rates as unconstrained firms (Panel a), yet experience sharper wage premium declines (Panel b), consistent with underexploitation of the intangible-skill complementarity.}
\end{figure}

This dynamic evidence completes the empirical case for skill-biased stagnation. Constrained firms respond to the skill supply shock by hiring skilled workers, but their inability to build complementary intangible capital means these workers are underexploited. The wage premium falls more sharply at constrained firms, revealing that financial frictions prevent the economy from fully exploiting productivity gains from rising human capital.

\subsection{Summary of Empirical Evidence}

The three facts establish the key building blocks of the theoretical mechanism. First, technology features complementarity between intangibles and skills. Second, financial frictions distort investment composition toward tangibles and away from intangibles. Third, the combination of these forces prevents constrained firms from exploiting complementarity in both the cross-section and over time: the interaction coefficient is weak for constrained firms, and wage premia declined more sharply at constrained firms during the skill supply expansion of 2011-2022. The following sections develop a model that rationalizes these patterns and enables quantitative counterfactual analysis.

\section{Building Intuition: A Toy Model}
\label{sec:toy_model}

\subsection{Environment}

Consider a static economy with a representative firm and no aggregate uncertainty. Within the period, there are two sequential stages: an \emph{investment stage} followed by a \emph{production stage}. The firm is endowed with initial tangible capital $K_0 > 0$, initial wealth $W_0 \geq 0$, and a total allocation of skilled labor $\bar{H} > 0$. Productivity $z > 0$ is constant and known.

\textbf{Investment Stage:} The firm chooses tangible investment $I^K \geq 0$ and allocates skilled labor between R\&D ($H^R \geq 0$) and production ($H^P = \bar{H} - H^R$). Intangible capital is produced linearly:
\begin{equation}
S = \Gamma H^R, \quad \Gamma > 0.
\label{eq:toy_intangible_prod}
\end{equation}

\textbf{Production Stage:} The firm produces output using accumulated capital and the skilled labor allocated in the investment stage. No further investment decisions are made.

\subsection{Technology}

\textbf{Capital Stocks.} The tangible capital stock entering production is:
\begin{equation}
K = K_0 + I^K.
\end{equation}

\textbf{Intangible-Skill Composite.} The intangible capital $S$ from the investment stage combines with the allocated production labor $H^P = \bar{H} - H^R$ to form:
\begin{equation}
Q(S, H^P) = \left[\omega S^{\rho} + (1-\omega)(H^P)^{\rho}\right]^{1/\rho}, \quad \omega \in (0,1), \, \rho < 0.
\label{eq:toy_Q}
\end{equation}
The elasticity of substitution $\sigma_Q = 1/(1-\rho) < 1$ captures complementarity.

\textbf{Production.} Output in the production stage is:
\begin{equation}
Y = z K^{\alpha} Q(S, H^P)^{1-\alpha}, \quad \alpha \in (0,1).
\label{eq:toy_production}
\end{equation}

For tractability, I abstract from unskilled labor and normalize away depreciation.

\subsection{Financial Constraint}

The firm may borrow only against tangible assets. At the end of the investment stage, the pledgeable capital stock equals $K_0 + I^K$. Thus:
\begin{equation}
D \leq \alpha_K (K_0 + I^K), \quad \alpha_K \in (0,1),
\label{eq:toy_collateral}
\end{equation}
where $D$ denotes borrowing in the investment stage. Intangible capital is non-pledgeable: $\alpha_S = 0$.

\textbf{Investment Stage Budget Constraint:}
\begin{equation}
I^K + w_H H^R \leq W_0 + D.
\label{eq:toy_budget_raw}
\end{equation}

If the constraint binds, $D = \alpha_K(K_0 + I^K)$, so:
\begin{align}
I^K + w_H H^R &\leq W_0 + \alpha_K(K_0 + I^K), \\
I^K(1 - \alpha_K) + w_H H^R &\leq W_0 + \alpha_K K_0.
\label{eq:toy_budget}
\end{align}

Define \emph{effective wealth}:
\[
W'_0 \equiv W_0 + \alpha_K K_0,
\]
so the constraint becomes:
\begin{equation}
I^K(1 - \alpha_K) + w_H H^R \leq W'_0.
\end{equation}

\textbf{Interpretation.} Each additional unit of tangible investment increases borrowing capacity by $\alpha_K$, effectively reducing its cost to $(1-\alpha_K)$. R\&D yields no such benefit. The firm's effective resources include its financial wealth $W_0$ plus the borrowing capacity from its initial capital stock ($\alpha_K K_0$).

\subsection{Firm Problem}

The firm chooses $(I^K, H^R)$ in the investment stage to maximize net output:
\begin{equation}
\max_{I^K \geq 0, \, H^R \in [0, \bar{H}]} \quad Y - I^K - w_H \bar{H}
\label{eq:toy_objective}
\end{equation}
subject to the budget constraint \eqref{eq:toy_budget}, where:
\begin{align}
Y &= z(K_0 + I^K)^{\alpha} Q(\Gamma H^R, \bar{H} - H^R)^{1-\alpha}, \\
Q(\Gamma H^R, \bar{H} - H^R) &= \left[\omega(\Gamma H^R)^{\rho} + (1-\omega)(\bar{H} - H^R)^{\rho}\right]^{1/\rho}.
\end{align}

Note that total skilled labor cost $w_H \bar{H}$ is fixed, so the firm's choice concerns only the allocation of $\bar{H}$ between R\&D and production.

\subsubsection{Unconstrained Solution}

When $W'_0$ is sufficiently large that constraint \eqref{eq:toy_budget} does not bind, the first-order conditions are:
\begin{align}
\frac{\partial Y}{\partial I^K} &= \alpha z K^{\alpha - 1} Q^{1-\alpha} = 1, \label{eq:toy_foc_K_uncon} \\
\frac{\partial Y}{\partial H^R} &= (1-\alpha) z K^{\alpha} Q^{-\alpha} \left[\frac{\partial Q}{\partial S} \Gamma - \frac{\partial Q}{\partial H^P}\right] = 0. \label{eq:toy_foc_HR_uncon}
\end{align}

From \eqref{eq:toy_foc_HR_uncon}, the optimal allocation balances the marginal product of intangibles (via R\&D) against the marginal product of direct production labor:
\begin{equation}
\frac{\partial Q}{\partial S} \Gamma = \frac{\partial Q}{\partial H^P}.
\label{eq:toy_balance}
\end{equation}

This is an interior solution exploiting the complementarity structure \eqref{eq:toy_Q}.

\subsubsection{Constrained Solution}

When constraint \eqref{eq:toy_budget} binds, form the Lagrangian:
\begin{equation}
\mathcal{L} = Y - I^K - w_H \bar{H} - \mu[I^K(1-\alpha_K) + w_H H^R - W'_0],
\end{equation}
where $\mu > 0$ is the shadow value of resources.

The first-order conditions become:
\begin{align}
\frac{\partial Y}{\partial I^K} &= 1 + \mu(1 - \alpha_K), \label{eq:toy_foc_K_con} \\
\frac{\partial Y}{\partial H^R} &= \mu w_H. \label{eq:toy_foc_HR_con}
\end{align}

\textbf{Key Observation.} Comparing \eqref{eq:toy_foc_K_uncon} with \eqref{eq:toy_foc_K_con}: the constrained firm requires a higher marginal product of capital, implying lower $I^K$.

Comparing \eqref{eq:toy_foc_HR_uncon} with \eqref{eq:toy_foc_HR_con}: the constrained firm requires a positive marginal product of R\&D labor (instead of zero), implying lower $H^R$ (and correspondingly higher $H^P$) than the unconstrained firm.

\subsection{Pecking-Order Distortion}

Before stating the main proposition, I establish that the constrained firm produces a lower intangible-skill composite.

\begin{lemma}[Suboptimal Allocation]
\label{lem:toy_Q_lower}
The constrained firm produces a lower intangible-skill composite than the unconstrained firm: $Q_c < Q_u$.
\end{lemma}

\begin{proof}
Define $\tilde{Q}(H^R; \bar{H}) \equiv Q(\Gamma H^R, \bar{H} - H^R)$ as the intangible-skill composite viewed as a function of the R\&D allocation for given $\bar{H}$. The unconstrained firm chooses $H^R_u$ to satisfy \eqref{eq:toy_balance}, which is equivalent to maximizing $\tilde{Q}$ over $H^R \in [0, \bar{H}]$:
\[
H^R_u = \argmax_{H^R \in [0,\bar{H}]} \tilde{Q}(H^R; \bar{H}).
\]
The constrained firm chooses $H^R_c < H^R_u$ (as established below in Proposition \ref{prop:toy_pecking_order}). Since $\tilde{Q}$ is strictly concave in $H^R$ and $H^R_c \neq H^R_u$, we have $\tilde{Q}(H^R_c; \bar{H}) < \tilde{Q}(H^R_u; \bar{H})$, i.e., $Q_c < Q_u$.
\end{proof}

\begin{remark}[Parameter Dependence and Limiting Cases]
\label{rem:limiting_cases}
The result $Q_c < Q_u$ and its magnitude depend critically on the elasticity parameter $\rho$. The proof requires strict concavity of $\tilde{Q}(H^R) = [\omega(\Gamma H^R)^{\rho} + (1-\omega)(\bar{H} - H^R)^{\rho}]^{1/\rho}$, which holds for $\rho < 1$ with an interior optimum. We examine the limiting cases:

\textbf{Case 1: $\rho \to -\infty$ (Leontief, $\sigma_Q \to 0$).} In this limit,
\[
Q \to \min\{S, H^P\}
\]
after appropriate normalization of $\omega$. The composite becomes $\tilde{Q}(H^R) = \min\{\Gamma H^R, \bar{H} - H^R\}$, which is maximized at the balanced allocation $H^R_u = \bar{H}/(1 + \Gamma)$, yielding $Q_u = \Gamma\bar{H}/(1+\Gamma)$. Any deviation creates complete bottlenecks: the constrained firm's underinvestment in $S$ implies $Q_c = S_c = \Gamma H^R_c < \Gamma H^R_u \leq Q_u$, with $H^P_c$ entirely wasted beyond the binding constraint. The distortion is maximally severe.

\textbf{Case 2: $\rho \to 0$ (Cobb-Douglas, $\sigma_Q \to 1$).} Taking the limit via L'H\^opital's rule,
\[
Q \to S^{\omega}(H^P)^{1-\omega}.
\]
The composite $\tilde{Q}(H^R) = \Gamma^{\omega}(H^R)^{\omega}(\bar{H} - H^R)^{1-\omega}$ is strictly concave with interior maximum at $H^R_u = \omega\bar{H}$. The result $Q_c < Q_u$ holds, with the gap determined by the curvature $\tilde{Q}''(H^R_u) = -\omega(1-\omega)\Gamma^{\omega}\bar{H}^{-1}(H^R_u)^{\omega-2}(\bar{H}-H^R_u)^{-1-\omega} < 0$.

\textbf{Case 3: $\rho \in (0,1)$ (Gross Substitutes, $\sigma_Q > 1$).} The CES function remains strictly concave for $\rho < 1$, so the analysis proceeds unchanged. The composite $\tilde{Q}$ has a unique interior maximum, and $Q_c < Q_u$ holds. However, because $S$ and $H^P$ are substitutes, the constrained firm can partially compensate for low $S$ by allocating more $H^P$, reducing the magnitude of $Q_u - Q_c$ relative to the complementary case.

\textbf{Case 4: $\rho \to 1$ (Perfect Substitutes, $\sigma_Q \to \infty$).} In this limit,
\[
Q \to \omega S + (1-\omega)H^P,
\]
and the composite becomes $\tilde{Q}(H^R) = \omega\Gamma H^R + (1-\omega)(\bar{H} - H^R) = [\omega\Gamma - (1-\omega)]H^R + (1-\omega)\bar{H}$. This is \emph{linear} in $H^R$, so $\tilde{Q}$ is no longer strictly concave. Three subcases arise:
\begin{itemize}
\item If $\omega\Gamma > 1-\omega$: corner solution $H^R_u = \bar{H}$ (all labor to R\&D).
\item If $\omega\Gamma < 1-\omega$: corner solution $H^R_u = 0$ (all labor to production).
\item If $\omega\Gamma = 1-\omega$: the firm is indifferent over all allocations, and $Q = (1-\omega)\bar{H}$ regardless of $H^R$.
\end{itemize}
In the knife-edge case $\omega\Gamma = 1-\omega$, the financial constraint creates no distortion in $Q$ (though it still distorts $K$). Otherwise, the constraint may or may not bind depending on which corner is optimal.

\textbf{Summary.} The baseline assumption $\rho < 0$ ensures: (i) strict concavity of $\tilde{Q}$ with interior optimum, (ii) the result $Q_c < Q_u$ for any constrained allocation, and (iii) economically significant complementarity that amplifies the distortion. As $\rho \to -\infty$, the distortion becomes arbitrarily severe; as $\rho \to 1$, the distortion vanishes in the knife-edge case or the problem degenerates to corner solutions.
\end{remark}

\begin{proposition}[Pecking Order]
\label{prop:toy_pecking_order}
Under the collateral constraint \eqref{eq:toy_collateral} with $\alpha_K \in (0,1)$ and $\alpha_S = 0$, constrained firms ($\mu > 0$) exhibit:

\begin{enumerate}[label=(\roman*)]
\item Lower R\&D investment: $H^R_c < H^R_u$
\item Lower intangible capital: $S_c < S_u$
\item Higher production labor: $H^P_c > H^P_u$
\item Lower tangible investment: $I^K_c < I^K_u$
\end{enumerate}

where subscripts $c$ and $u$ denote constrained and unconstrained firms respectively.
\end{proposition}

\begin{proof}
\textbf{Parts (i)-(iii):} From \eqref{eq:toy_foc_HR_con}, the constrained firm requires $\partial Y/\partial H^R = \mu w_H > 0$, while the unconstrained firm has $\partial Y/\partial H^R = 0$ at optimum. Since
\[
\frac{\partial Y}{\partial H^R} = (1-\alpha) z K^{\alpha} Q^{-\alpha} \left[\frac{\partial Q}{\partial S} \Gamma - \frac{\partial Q}{\partial H^P}\right],
\]
and the term $(1-\alpha) z K^{\alpha} Q^{-\alpha} > 0$, the constrained firm has
\[
\frac{\partial Q}{\partial S} \Gamma > \frac{\partial Q}{\partial H^P}
\]
at its optimum, whereas the unconstrained firm has equality. Computing the partial derivatives of \eqref{eq:toy_Q}:
\[
\frac{\partial Q}{\partial S} = \omega S^{\rho-1} Q^{1-\rho}, \qquad \frac{\partial Q}{\partial H^P} = (1-\omega)(H^P)^{\rho-1} Q^{1-\rho}.
\]
The condition $\frac{\partial Q}{\partial S} \Gamma > \frac{\partial Q}{\partial H^P}$ implies a lower $S/H^P$ ratio than the unconstrained optimum. Given the constraint $H^P = \bar{H} - H^R$ and $S = \Gamma H^R$, this requires $H^R_c < H^R_u$. Parts (ii) and (iii) follow immediately.

\textbf{Part (iv):} Define $f(I^K; Q) \equiv \alpha z (K_0 + I^K)^{\alpha-1} Q^{1-\alpha}$, which is decreasing in $I^K$ (since $\alpha - 1 < 0$) and increasing in $Q$.

The unconstrained firm satisfies $f(I^K_u; Q_u) = 1$.

For the constrained firm, evaluate $f$ at $I^K_u$ with $Q_c$:
\[
f(I^K_u; Q_c) = \alpha z (K_0 + I^K_u)^{\alpha-1} Q_c^{1-\alpha} < \alpha z (K_0 + I^K_u)^{\alpha-1} Q_u^{1-\alpha} = 1,
\]
where the inequality uses $Q_c < Q_u$ (Lemma \ref{lem:toy_Q_lower}) and $(1-\alpha) > 0$.

The constrained firm's FOC \eqref{eq:toy_foc_K_con} requires $f(I^K_c; Q_c) = 1 + \mu(1-\alpha_K) > 1$. Since $f(\cdot; Q_c)$ is strictly decreasing in $I^K$ and $f(I^K_u; Q_c) < 1 < f(I^K_c; Q_c)$, we must have $I^K_c < I^K_u$.
\end{proof}

\begin{remark}
The asymmetry arises from differential collateral value. Each unit of $I^K$ reduces the effective resource constraint by $(1-\alpha_K)$ because it generates $\alpha_K$ in borrowing capacity. In contrast, $H^R$ provides no collateral benefit since $\alpha_S = 0$. This creates an implicit subsidy for tangible investment relative to R\&D.
\end{remark}

\subsection{Underexploitation of Complementarity}

\begin{lemma}[Complementarity]
\label{lem:toy_complementarity}
The intangible-skill composite \eqref{eq:toy_Q} with $\rho < 0$ satisfies:
\begin{equation}
\frac{\partial^2 Q}{\partial S \, \partial H^P} = \omega(1-\omega)(1-\rho) S^{\rho-1} (H^P)^{\rho-1} Q^{1-2\rho} > 0.
\end{equation}
\end{lemma}

\begin{proof}
Direct differentiation of \eqref{eq:toy_Q}. All terms are positive when $\rho < 0$ since $(1-\rho) > 0$ and $(1-2\rho) > 0$.
\end{proof}

\begin{corollary}[Underexploitation of Skilled Labor]
\label{cor:toy_underexploitation}
Despite having higher production labor ($H^P_c > H^P_u$), the constrained firm has lower marginal product of that labor:
\begin{equation}
\frac{\partial Q}{\partial H^P}(S_c, H^P_c) < \frac{\partial Q}{\partial H^P}(S_u, H^P_u).
\end{equation}
\end{corollary}

\begin{proof}
The marginal product of production labor can be rewritten as:
\begin{equation}
\frac{\partial Q}{\partial H^P} = (1-\omega)(H^P)^{\rho-1} Q^{1-\rho} = (1-\omega) \left(\frac{Q}{H^P}\right)^{1-\rho}.
\label{eq:MP_HP_ratio}
\end{equation}
Since $\rho < 0$, we have $1 - \rho > 1$, so \eqref{eq:MP_HP_ratio} is strictly increasing in the ratio $Q/H^P$.

From Lemma \ref{lem:toy_Q_lower}, $Q_c < Q_u$. From Proposition \ref{prop:toy_pecking_order}(iii), $H^P_c > H^P_u$. Therefore:
\[
\frac{Q_c}{H^P_c} < \frac{Q_u}{H^P_u},
\]
which implies $\frac{\partial Q}{\partial H^P}(S_c, H^P_c) < \frac{\partial Q}{\partial H^P}(S_u, H^P_u)$.
\end{proof}

\textbf{Economic Interpretation.} The constrained firm overallocates skilled labor to production ($H^P$) relative to intangibles ($S$) because of the financial distortion. However, due to complementarity, the productivity of that labor is limited by the low stock of intangibles. This creates a \emph{double inefficiency}: wrong allocation \emph{and} underexploitation of the resources allocated to production.

\subsection{Skill-Biased Stagnation}

Consider a comparative static exercise: an increase in the skilled labor endowment $\bar{H}$.

\subsubsection{Unconstrained Firm Response}

For the unconstrained firm, the optimality condition \eqref{eq:toy_balance} determines the allocation of $\bar{H}$ between $H^R$ and $H^P$. Using the expressions for marginal products:
\[
\omega \Gamma (\Gamma H^R)^{\rho-1} = (1-\omega)(\bar{H} - H^R)^{\rho-1}.
\]
This condition pins down the \emph{ratio} $H^R/(\bar{H} - H^R)$ independently of $\bar{H}$. Therefore, the R\&D share $\phi^R \equiv H^R/\bar{H}$ is constant:
\begin{equation}
\frac{dH^R_u}{d\bar{H}} = \phi^R_u, \qquad \frac{dH^P_u}{d\bar{H}} = 1 - \phi^R_u.
\label{eq:unconstrained_response}
\end{equation}

Both margins expand proportionally, fully exploiting the complementarity.

\subsubsection{Constrained Firm Response}

The binding constraint implies:
\[
I^K_c(1-\alpha_K) + w_H H^R_c = W'_0.
\]
Differentiating:
\[
(1-\alpha_K)\,dI^K_c + w_H\, dH^R_c = 0.
\]

Because both constrained FOCs depend on 
\[
Q_c = Q(\Gamma H^R_c, \bar{H} - H^R_c),
\]
which varies with $\bar{H}$, the pair $(I^K_c, H^R_c)$ generally adjusts with $\bar{H}$.

However, the constraint imposes a strong wedge: additional skilled labor primarily enters production. Formally, the constrained R\&D response satisfies:
\begin{equation}
0 \le \frac{dH^R_c}{d\bar{H}} < \phi^R_u, \qquad \frac{dH^P_c}{d\bar{H}} = 1 - \frac{dH^R_c}{d\bar{H}}.
\label{eq:constrained_response}
\end{equation}

\begin{proposition}[Skill-Biased Stagnation]
\label{prop:toy_stagnation}
When the skilled labor endowment increases ($\bar{H} \uparrow$):
\begin{enumerate}[label=(\roman*)]
\item The unconstrained firm increases both $H^R$ and $H^P$ proportionally, building higher $S$ and exploiting complementarity.
\item The constrained firm expands $H^P$ strictly more than $H^R$:
\[
0 \le \frac{dH^R_c}{d\bar{H}} < \phi^R_u.
\]
\item The intangible-skill composite grows faster for the unconstrained firm:
\begin{equation}
\frac{dQ_u}{d\bar{H}} > \frac{dQ_c}{d\bar{H}} > 0.
\label{eq:dQ_comparison}
\end{equation}
\item The output gap between firms widens:
\begin{equation}
\frac{d(Y_u - Y_c)}{d\bar{H}} > 0.
\label{eq:output_gap_widens}
\end{equation}
\end{enumerate}
\end{proposition}

\begin{proof}
Parts (i)-(ii) follow from the derived responses.

\textbf{Part (iii):} By the envelope theorem, since the unconstrained firm maximizes $Q$ over the allocation of $\bar{H}$:
\[
\frac{dQ_u}{d\bar{H}} = \frac{\partial Q}{\partial H^P}\bigg|_{(S_u, H^P_u)}.
\]
For the constrained firm:
\[
\frac{dQ_c}{d\bar{H}} = \frac{\partial Q}{\partial S}\Gamma \frac{dH^R_c}{d\bar{H}} + \frac{\partial Q}{\partial H^P}\left(1 - \frac{dH^R_c}{d\bar{H}}\right).
\]
Since the unconstrained firm satisfies $\frac{\partial Q}{\partial S}\Gamma = \frac{\partial Q}{\partial H^P}$, we can write:
\[
\frac{dQ_u}{d\bar{H}} = \frac{\partial Q}{\partial H^P}\bigg|_{(S_u, H^P_u)} = \frac{\partial Q}{\partial S}\bigg|_{(S_u, H^P_u)}\Gamma = \left[\frac{\partial Q}{\partial S}\Gamma \phi^R_u + \frac{\partial Q}{\partial H^P}(1-\phi^R_u)\right]_{(S_u, H^P_u)}.
\]

From Corollary \ref{cor:toy_underexploitation}, $\frac{\partial Q}{\partial H^P}(S_u, H^P_u) > \frac{\partial Q}{\partial H^P}(S_c, H^P_c)$. Moreover, from the constrained firm's FOC, $\frac{\partial Q}{\partial S}(S_c, H^P_c)\Gamma > \frac{\partial Q}{\partial H^P}(S_c, H^P_c)$. Since $\frac{dH^R_c}{d\bar{H}} < \phi^R_u$ (part ii), the constrained firm places less weight on the higher-valued margin (R\&D) and more weight on the lower-valued margin (production), establishing \eqref{eq:dQ_comparison}.

\textbf{Part (iv):} From $Y = zK^{\alpha}Q^{1-\alpha}$:
\[
\frac{dY}{d\bar{H}} = (1-\alpha) z K^{\alpha} Q^{-\alpha} \frac{dQ}{d\bar{H}} = (1-\alpha) \frac{Y}{Q} \frac{dQ}{d\bar{H}}.
\]

Define the ``output-to-composite'' ratio $\psi \equiv Y/Q = zK^{\alpha}Q^{-\alpha}$. Then:
\[
\frac{d(Y_u - Y_c)}{d\bar{H}} = (1-\alpha)\left[\psi_u \frac{dQ_u}{d\bar{H}} - \psi_c \frac{dQ_c}{d\bar{H}}\right].
\]

From part (iii), $\frac{dQ_u}{d\bar{H}} > \frac{dQ_c}{d\bar{H}}$. Using \eqref{eq:MP_HP_ratio}:
\[
\frac{dQ}{d\bar{H}} \ge \frac{\partial Q}{\partial H^P} = (1-\omega)\left(\frac{Q}{H^P}\right)^{1-\rho}.
\]

Therefore:
\begin{align}
\frac{dY}{d\bar{H}} &\ge (1-\alpha) z K^{\alpha} Q^{-\alpha} (1-\omega)\left(\frac{Q}{H^P}\right)^{1-\rho} \nonumber \\
&= (1-\alpha)(1-\omega) z K^{\alpha} Q^{1-\alpha-\rho} (H^P)^{-(1-\rho)}.
\label{eq:dY_dH_expanded}
\end{align}

From Proposition \ref{prop:toy_pecking_order}: $K_u > K_c$, $Q_u > Q_c$, and $H^P_u < H^P_c$. Since $\alpha > 0$, $1-\alpha-\rho > 0$ (as $\rho < 0$), and $1-\rho > 0$:
\begin{itemize}
\item $K_u^{\alpha} > K_c^{\alpha}$
\item $Q_u^{1-\alpha-\rho} > Q_c^{1-\alpha-\rho}$
\item $(H^P_u)^{-(1-\rho)} > (H^P_c)^{-(1-\rho)}$
\end{itemize}
All three factors favor the unconstrained firm, so $\frac{dY_u}{d\bar{H}} > \frac{dY_c}{d\bar{H}}$, establishing \eqref{eq:output_gap_widens}.
\end{proof}

\textbf{Economic Intuition.} The skill-biased stagnation arises because constrained firms cannot finance the R\&D needed to build intangibles that would complement the additional skilled labor in production. While they can deploy more $H^P$, the lack of complementary $S$ means this labor is underexploited. Unconstrained firms, by contrast, expand both margins and realize the full gains from complementarity.

\newpage
\section{Quantitative Model}
\label{sec:quantitative_model}

\subsection{Environment}

\subsubsection{Time and Agents}
Time is discrete. A continuum of firms $j \in [0,1]$ faces idiosyncratic productivity $z_{j,t}$. Each firm holds tangible capital $K_{j,t}$, intangible capital $S_{j,t}$, and debt $D_{j,t}$. Firms exit with probability $\zeta \in (0,1)$; entrants draw $z_0 \sim F_z$ on $[\underline{z}, \overline{z}]$ and start with $K_0 = S_0 = D_0 = 0$, receiving initial equity $a_0 > 0$ financed by household transfers.

\subsubsection{Household}
A representative household owns all firms, supplies skilled and unskilled labor inelastically, and has standard preferences over consumption:
\[
\E_0 \sum_{t=0}^{\infty} \beta^t u(C_t), \qquad 0 < \beta < 1.
\]
The household supplies $\bar{L}$ units of unskilled labor and $\bar{H}$ units of skilled labor, where $\bar{L} + \bar{H} = 1$.

The household's budget constraint is:
\begin{equation}
C_t + B_{t+1} + T_t = w_L \bar{L} + w_H \bar{H} + (1+r) B_t + \Pi_t,
\label{eq:hh_budget}
\end{equation}
where $B_t$ denotes household deposits at banks, $\Pi_t$ is aggregate firm dividends, and $T_t$ represents transfers to finance new entrants replacing exiting firms.

The household's Euler equation with linear utility pins down:
\begin{equation}
1 + r = \frac{1}{\beta}.
\label{eq:interest_rate}
\end{equation}

\subsubsection{Financial Intermediation}
Competitive banks accept deposits from households at rate $r$ and lend to firms at the same rate. Banks perfectly enforce repayment up to collateral value and make zero profits. No default occurs in equilibrium---the collateral constraint binds ex ante, preventing default, following \citet{kiyotaki1997credit}.

\subsection{Technology}

\subsubsection{Idiosyncratic Productivity}
\[
\log z_{j,t+1} = \rho_z \log z_{j,t} + \sigma_z \varepsilon_{j,t+1}, \quad \varepsilon_{j,t+1} \sim N(0,1),
\]
with $0 < \rho_z < 1$, $\sigma_z > 0$.

\subsubsection{Production Technology}

Production involves a nested CES structure. First, define the capital composite:
\begin{equation}
X_{j,t} = \Big[\theta_K K_{j,t}^{\rho_K} + \theta_Q Q_{j,t}^{\rho_K}\Big]^{1/\rho_K},
\label{eq:capital_composite}
\end{equation}
where $Q_{j,t}$ is an intangible-skill bundle defined as:
\begin{equation}
Q_{j,t} = \Big[\omega S_{j,t}^{\rho_Q} + (1-\omega)(H^P_{j,t})^{\rho_Q}\Big]^{1/\rho_Q}.
\label{eq:intangible_skill_bundle}
\end{equation}

The final production function is:
\begin{equation}
Y_{j,t} = z_{j,t} \Big[X_{j,t}^{\alpha} L_{j,t}^{\gamma}\Big]^{\nu},
\label{eq:prod}
\end{equation}
where $\alpha, \gamma, \nu \in (0,1)$, $\theta_K, \theta_Q, \omega \in (0,1)$, and $\alpha + \gamma = 1$ (constant returns in the inner nest).

\begin{assumption}[Complementarity Structure]
\label{ass:complementarity}
Define elasticities $\sigma_K \equiv 1/(1-\rho_K)$ and $\sigma_Q \equiv 1/(1-\rho_Q)$. We assume $\sigma_Q < \sigma_K$ (i.e., $\rho_Q < \rho_K < 0$), so that intangibles and skilled labor are stronger complements than tangible capital and the intangible-skill bundle.
\end{assumption}

\begin{remark}
Assumption \ref{ass:complementarity} captures the idea that intangible capital and skilled labor work together in a more integrated way than tangible capital does with the intangible-skill composite $Q$. Empirical evidence from \citet{gozenozkara2024} supports the existence of synergies between intangibles and skilled labor.
\end{remark}

\subsubsection{Capital Accumulation}

Tangible capital evolves via standard accumulation:
\begin{equation}
K_{j,t+1} = (1-\delta_K) K_{j,t} + I^K_{j,t},
\label{eq:capital_accum_K}
\end{equation}
where $I^K_{j,t}$ is tangible investment and $0 < \delta_K < 1$.

Intangible capital is produced via R\&D labor:
\begin{equation}
S_{j,t+1} = (1-\delta_S) S_{j,t} + \Gamma (H^R_{j,t})^{\xi}, \qquad 0 < \xi \le 1,
\label{eq:capital_accum_S}
\end{equation}
where $H^R_{j,t}$ is skilled labor allocated to R\&D, $\Gamma > 0$, and $\xi \le 1$ captures (weakly) decreasing returns to R\&D labor in knowledge creation.\footnote{The case $\xi < 1$ captures diminishing returns in R\&D production. The limiting case $\xi = 1$ corresponds to linear intangible production, as in \citet{atkeson2010innovation}.}

Total skilled labor hired by the firm satisfies:
\begin{equation}
H_{j,t} = H^P_{j,t} + H^R_{j,t}.
\label{eq:skilled_labor_total}
\end{equation}

\subsubsection{Adjustment Costs}

Following the investment literature \citep{cooper2006dynamics, khan2008idiosyncratic}, capital adjustment is subject to convex costs:
\begin{equation}
\Phi^K_{j,t} = \frac{\phi_K}{2} \frac{(I^K_{j,t})^2}{K_{j,t}}, \qquad 
\Phi^S_{j,t} = \frac{\phi_S}{2} \frac{(\Delta S_{j,t})^2}{S_{j,t}},
\label{eq:adj_costs}
\end{equation}
where $\Delta S_{j,t} \equiv S_{j,t+1} - (1-\delta_S)S_{j,t} = \Gamma(H^R_{j,t})^{\xi}$ is gross intangible investment. Setting $\phi_K = \phi_S = 0$ nests the frictionless case.

\subsection{Financial Frictions}

\subsubsection{Collateral Constraint}

Following the costly-state-verification literature \citep{townsend1979optimal, bernanke1999financial}, lenders can recover only a fraction of firm assets in case of default. Define recovery rates:
\begin{equation}
\alpha_K \in (0,1), \qquad \alpha_S \in [0, \alpha_K),
\label{eq:recovery_rates}
\end{equation}
where $\alpha_K$ is the recovery rate on tangible capital and $\alpha_S$ is the recovery rate on intangible capital, with $\alpha_S < \alpha_K$ reflecting the lower pledgeability of intangibles \citep{holttinenmelolinnafroem2025}.

\begin{assumption}[Low Intangible Pledgeability]
\label{ass:pledgeability}
Intangible assets are less pledgeable than tangible assets: $0 \le \alpha_S < \alpha_K < 1$. The baseline calibration considers $\alpha_S = 0$ (intangibles fully non-pledgeable).
\end{assumption}

To ensure no default in equilibrium, banks lend only up to recoverable collateral value. Following the timing convention in \citet{kiyotaki1997credit} and \citet{jermann2012macroeconomic}, debt $D_{j,t+1}$ taken in period $t$ (repaid in $t+1$) is collateralized by capital available when repayment is due:
\begin{equation}
D_{j,t+1} \le \alpha_K K_{j,t+1} + \alpha_S S_{j,t+1}.
\label{eq:collateral_constraint}
\end{equation}
Since $K_{j,t+1}$ and $S_{j,t+1}$ depend on current investment decisions, this creates a direct link between investment and borrowing capacity within the period.

\subsubsection{Timing}

The stages below describe the sequence of decisions within period $t$:

\begin{enumerate}[leftmargin=*]
\item \textbf{Beginning of period:} Firm enters with state $(z_{t-1}, K_t, S_t, D_t)$, where $D_t$ is debt to be repaid this period.

\item \textbf{Productivity realization:} Current productivity $z_t$ is drawn from the AR(1) process.

\item \textbf{Exit shock:} With probability $\zeta$, the firm exits: it produces, collects revenue net of wages, liquidates capital, repays debt $(1+r)D_t$, and distributes residual value to shareholders. With probability $(1-\zeta)$, the firm survives.

\item \textbf{Production decisions:} Surviving firms choose labor $(L_t, H^P_t)$ to maximize static profits.

\item \textbf{Investment and financing:} Firms choose tangible investment $I^K_t$, R\&D labor $H^R_t$, and new debt $D_{t+1}$ subject to the collateral constraint.

\item \textbf{Dividend payment:} Firms pay dividends $\text{Div}_t \ge 0$ to shareholders.

\item \textbf{End of period:} Firm enters $t+1$ with state $(z_t, K_{t+1}, S_{t+1}, D_{t+1})$.
\end{enumerate}

\textbf{Timing and Wage Equality.} Both $H^P$ (production) and $H^R$ (R\&D) are skilled labor employed within period $t$ at wage $w_H$. The single wage clears the aggregate skilled labor market: $\int (H^P_j + H^R_j) d\Psi = \bar{H}$.

\subsubsection{Budget Constraint and Dividends}

The firm's budget constraint equates sources and uses of funds:
\begin{equation}
I^K_{j,t} + \Phi^K_{j,t} + \Phi^S_{j,t} + w_L L_{j,t} + w_H (H^P_{j,t} + H^R_{j,t}) + (1+r) D_{j,t} = Y_{j,t} + D_{j,t+1} + \text{Div}_{j,t}.
\label{eq:budget_constraint}
\end{equation}

Define gross profits from production:
\begin{equation}
\Pi^{gross}_{j,t} \equiv Y_{j,t} - w_L L_{j,t} - w_H H^P_{j,t}.
\label{eq:gross_profits_def}
\end{equation}

Dividends distributed to shareholders are:
\begin{equation}
\text{Div}_{j,t} = \Pi^{gross}_{j,t} - I^K_{j,t} - \Phi^K_{j,t} - \Phi^S_{j,t} - w_H H^R_{j,t} - (1+r) D_{j,t} + D_{j,t+1}.
\label{eq:dividends}
\end{equation}

Firms cannot issue new equity:
\begin{equation}
\text{Div}_{j,t} \ge 0.
\label{eq:div_constraint}
\end{equation}

\subsection{Firm Problem}

\subsubsection{Static Production Problem}

Given capital $(K, S)$ and productivity $z$, the firm chooses labor $(L, H^P)$ to maximize static profits:
\begin{equation}
\pi(z, K, S) = \max_{L, H^P} \Big\{ Y(z, K, S, L, H^P) - w_L L - w_H H^P \Big\}.
\label{eq:profit_function}
\end{equation}

The first-order conditions are:
\begin{equation}
\frac{\partial Y}{\partial L} = w_L, \qquad \frac{\partial Y}{\partial H^P} = w_H.
\label{eq:foc_labor}
\end{equation}
These yield policy functions $L^*(z, K, S)$ and $H^{P*}(z, K, S)$. Labor decisions are undistorted by financial frictions since labor is paid contemporaneously.

\subsubsection{Dynamic Problem}

The firm maximizes the expected present value of dividends. The state is $(z, K, S, D)$, and the value function satisfies:
\begin{equation}
V(z, K, S, D) = \max_{D', I^K, H^R} \Big\{ \text{Div} + \beta(1-\zeta) \E[V(z', K', S', D') \mid z] \Big\},
\label{eq:value_function}
\end{equation}
subject to:
\begin{itemize}
\item Dividends: $\text{Div} = \pi(z, K, S) - I^K - \Phi^K - \Phi^S - w_H H^R - (1+r) D + D'$

\item Capital accumulation: $K' = (1-\delta_K) K + I^K$, \quad $S' = (1-\delta_S) S + \Gamma (H^R)^{\xi}$

\item Collateral constraint: $D' \le \alpha_K K' + \alpha_S S'$

\item Non-negative dividends: $\text{Div} \ge 0$
\end{itemize}

\subsubsection{First-Order Conditions}

Let $\mu \ge 0$ denote the Lagrange multiplier on the dividend constraint \eqref{eq:div_constraint}, and $\lambda \ge 0$ the multiplier on the collateral constraint \eqref{eq:collateral_constraint}. When $\mu = 0$, dividends are positive and internal funds are not scarce; when $\mu > 0$, the dividend constraint binds and internal funds command a premium. When $\lambda = 0$, the collateral constraint is slack; when $\lambda > 0$, it binds.

\paragraph{FOC for Debt ($D'$).}
\begin{equation}
1 + \mu = \lambda + \beta(1-\zeta)(1+r) \E[(1 + \mu')].
\label{eq:foc_D}
\end{equation}
For an unconstrained firm ($\mu = \lambda = 0$), this reduces to $1 = \beta(1-\zeta)(1+r)$, which holds when exit risk exactly offsets the discount factor.

\paragraph{FOC for Tangible Investment ($I^K$).}
\begin{equation}
(1 + \mu)\left(1 + \phi_K \frac{I^K}{K}\right) = \lambda \alpha_K + \beta(1-\zeta) \E[V_{K'}].
\label{eq:foc_IK}
\end{equation}

The left-hand side is the marginal cost: purchasing one unit plus adjustment costs, valued at $(1+\mu)$ reflecting the shadow cost of internal funds. The right-hand side has two components:
\begin{enumerate}
\item $\lambda \alpha_K$: the immediate collateral benefit---investment increases $K'$, relaxing the current-period borrowing constraint.
\item $\beta(1-\zeta) \E[V_{K'}]$: the expected continuation value of additional capital.
\end{enumerate}

\paragraph{Envelope Condition for $K$.}
\begin{equation}
V_K = (1+\mu) \frac{\partial \pi}{\partial K} + (1+\mu) \frac{\phi_K}{2}\left(\frac{I^K}{K}\right)^2 + (1-\delta_K)\Big[\lambda \alpha_K + \beta(1-\zeta) \E[V_{K'}]\Big].
\label{eq:envelope_K}
\end{equation}

The term $(1+\mu) \frac{\phi_K}{2}(I^K/K)^2$ reflects that higher capital reduces adjustment costs for given investment. The collateral term $\lambda \alpha_K$ is multiplied by $(1-\delta_K)$ because only undepreciated capital carries forward.

Combining \eqref{eq:foc_IK} and \eqref{eq:envelope_K} yields the Euler equation:
\begin{align}
(1 + \mu_t)\left(1 + \phi_K \frac{I^K_t}{K_t}\right) &= \lambda_t \alpha_K + \beta(1-\zeta) \E_t\Bigg[ (1+\mu_{t+1}) \frac{\partial \pi_{t+1}}{\partial K_{t+1}} + (1+\mu_{t+1}) \frac{\phi_K}{2}\left(\frac{I^K_{t+1}}{K_{t+1}}\right)^2 \nonumber \\
&\quad + (1-\delta_K)\Big(\lambda_{t+1} \alpha_K + \beta(1-\zeta) \E_{t+1}[V_{K,t+2}]\Big)\Bigg].
\label{eq:euler_K}
\end{align}

\paragraph{FOC for Intangible Investment (via $H^R$).} Since $\partial S'/\partial H^R = \Gamma \xi (H^R)^{\xi-1}$:
\begin{equation}
(1+\mu)\left(w_H + \phi_S \frac{\Delta S}{S} \cdot \Gamma \xi (H^R)^{\xi-1}\right) = \Big[\lambda \alpha_S + \beta(1-\zeta) \E[V_{S'}]\Big] \cdot \Gamma \xi (H^R)^{\xi-1}.
\label{eq:foc_HR}
\end{equation}

Define the marginal cost of producing intangibles as $q_S \equiv w_H / [\Gamma \xi (H^R)^{\xi-1}]$. Abstracting from adjustment costs, the FOC simplifies to:
\begin{equation}
(1+\mu) \cdot q_S = \lambda \alpha_S + \beta(1-\zeta) \E[V_{S'}].
\label{eq:foc_S}
\end{equation}

\paragraph{Envelope Condition for $S$.}
\begin{equation}
V_S = (1+\mu) \frac{\partial \pi}{\partial S} + (1+\mu) \frac{\phi_S}{2}\left(\frac{\Delta S}{S}\right)^2 + (1-\delta_S)\Big[\lambda \alpha_S + \beta(1-\zeta) \E[V_{S'}]\Big].
\label{eq:envelope_S}
\end{equation}

\paragraph{Envelope Condition for $D$.}
\begin{equation}
V_D = -(1+\mu)(1+r).
\label{eq:envelope_D}
\end{equation}

\subsubsection{The Pecking-Order Distortion}

Comparing the FOCs for tangible \eqref{eq:foc_IK} and intangible \eqref{eq:foc_S} investment reveals the core mechanism.

\begin{proposition}[Pecking-Order Distortion]
\label{prop:pecking_order}
For a financially constrained firm ($\lambda > 0$), the immediate collateral benefit is larger for tangible than intangible investment:
\begin{equation}
\underbrace{\lambda \alpha_K}_{\text{collateral benefit of } K'} > \underbrace{\lambda \alpha_S}_{\text{collateral benefit of } S'}
\label{eq:pecking_order}
\end{equation}
since $\alpha_K > \alpha_S$.
\end{proposition}

\textbf{Intuition.} Each unit of tangible capital generates borrowing capacity $\alpha_K$, while intangible capital generates only $\alpha_S < \alpha_K$. When the collateral constraint binds ($\lambda > 0$), this asymmetry creates an implicit subsidy for tangible investment: the effective cost of tangible investment is reduced by $\lambda \alpha_K / (1+\mu)$, while intangible investment is reduced by only $\lambda \alpha_S / (1+\mu)$.

For unconstrained firms ($\lambda = 0$), the collateral terms vanish and investment depends only on marginal products---the first-best allocation.

\paragraph{Capital Composition Distortion.} Rearranging the FOCs in steady state:
\begin{equation}
\frac{\text{MB}_K - \lambda \alpha_K}{\text{MB}_S - \lambda \alpha_S} = \frac{(1+\mu)(1 + \phi_K I^K/K)}{(1+\mu) q_S},
\label{eq:capital_ratio}
\end{equation}
where $\text{MB}_K$ and $\text{MB}_S$ denote the continuation value terms. For constrained firms, since $\alpha_K > \alpha_S$, the numerator is reduced more than the denominator, distorting capital composition toward tangibles.

\subsection{Equilibrium}

\begin{definition}[Stationary Recursive Equilibrium]
\label{def:equilibrium}
A stationary recursive equilibrium consists of:
\begin{enumerate}[label=(\roman*)]
\item Firm value function $V(z, K, S, D)$ and policy functions $\{K'(\cdot), S'(\cdot), D'(\cdot), I^K(\cdot), H^R(\cdot), L(\cdot), H^P(\cdot)\}$,
\item Wages $(w_L, w_H)$ and interest rate $r$,
\item A stationary distribution of firms $\Psi^*(z, K, S, D)$,
\item Aggregate quantities $\{C, Y_{agg}, K_{agg}, S_{agg}, D_{agg}\}$,
\end{enumerate}
such that:
\begin{enumerate}[label=(\alph*)]
\item \textbf{Firms optimize:} $V(z, K, S, D)$ solves \eqref{eq:value_function} and policies satisfy the FOCs.

\item \textbf{Household optimizes:} $1 + r = 1/\beta$ from the Euler equation.

\item \textbf{Labor markets clear:}
\[
\int L_j \, d\Psi^* = \bar{L}, \qquad \int (H^P_j + H^R_j) \, d\Psi^* = \bar{H}.
\]

\item \textbf{Credit market clears:}
\[
B = \int D_j \, d\Psi^*.
\]

\item \textbf{Goods market clears:}
\[
C + \int \big(I^K_j + \Phi^K_j + \Phi^S_j\big) d\Psi^* + T = \int Y_j \, d\Psi^*,
\]
where $T = \zeta a_0$ finances entrants.

\item \textbf{Free entry:} The expected value of an entrant equals required initial equity:
\[
\int V(z, 0, 0, 0) \, dF_z(z) = a_0.
\]

\item \textbf{Stationarity:} $\Psi^*$ is invariant under the transition implied by policy functions, exit, and entry.
\end{enumerate}
\end{definition}

\section{Quantitative Experiments}
\label{sec:plan}

The quantitative model enables several counterfactual experiments to quantify the skill-biased stagnation mechanism.

\paragraph{Baseline Experiment: Skill Supply Shock.} I compute steady-state economies for different values of the skill ratio $\bar{H}/\bar{L}$, comparing outcomes with financial frictions (baseline calibration) against a frictionless counterfactual (setting $\lambda=0$ for all firms). This experiment measures how financial frictions mediate the aggregate productivity response to skill accumulation.

\paragraph{Transition Dynamics.} An alternative approach computes the full transition path following an unexpected permanent increase in $\bar{H}/\bar{L}$, tracking how constrained and unconstrained firms adjust their capital stocks and employment over time. This captures the dynamic misallocation channel as financially constrained firms slowly build intangible capital.

\paragraph{Financial Liberalization.} A third experiment varies the pledgeability of intangibles $\alpha_S$ while holding technology fixed, simulating financial innovations that improve the collateralizability of intangible assets. The baseline calibration uses $\alpha_S = 0.134$ and $\alpha_K = 0.381$ from \citet{holttinen2025aggregate}. Increasing $\alpha_S$ toward $\alpha_K$ quantifies the TFP gains from reducing the pledgeability gap. 

\clearpage
\appendix

\section{Data Construction and Cleaning}
\label{app:data_cleaning}

\subsection{Data Sources and Sample Construction}

The empirical analysis uses Portuguese firm-level data from two administrative sources for the period 2011-2022:

\begin{enumerate}[label=(\roman*)]
\item \textbf{SCIE (Sistema de Contas Integradas das Empresas):} Balance sheet and income statement data covering all Portuguese firms required to file annual accounts. Variables include assets, liabilities, equity, revenue, costs, investment flows, and R\&D expenditures.

\item \textbf{QP (Quadros de Pessoal):} Matched employer-employee data containing firm characteristics and worker-level information. I aggregate worker data to construct firm-level skill composition measures, defining skilled workers as those with tertiary education (ISCED 5-8).
\end{enumerate}

Before the cleaning procedure, I restrict the sample to private incorporated businesses (\textit{sociedades}), as balance sheet data is only available for incorporated firms and public firms face different objectives and constraints. The initial merged dataset (SCIE-matched, private incorporated firms) contains 2,329,807 firm-year observations spanning 2011-2022.

\subsection{Price Deflators}

All monetary variables are deflated to constant 2020 prices using three price indices:

\begin{itemize}
\item \textbf{GDP deflator} (base year 2020=100): Applied to revenue, costs, wages, and R\&D expenditures. Source: FRED (Federal Reserve Economic Data), series PRTGDPDEFQISMEI\_NBD20200101.

\item \textbf{GFCF deflator} (Gross Fixed Capital Formation, 2020=100): Applied to investment and disinvestment flows. Source: EU KLEMS-INTAN database, Portuguese data.

\item \textbf{Capital deflator} (2020=100): Applied to balance sheet stocks. Source: EU KLEMS-INTAN database, Portuguese data.
\end{itemize}

For 2022, GFCF and capital deflators are extrapolated using the 2021-2022 growth rate of the GDP deflator, as EU KLEMS-INTAN data availability ends in 2021.

\subsection{Data Cleaning Procedure}

The cleaning procedure follows a sequential pipeline to ensure data quality before constructing intangible capital stocks. All monetary variables are first deflated to constant 2020 prices using appropriate deflators: GDP deflator for revenue, costs, and wages; GFCF deflator for investment flows; and capital deflator for balance sheet stocks.

\paragraph{Step 1: Structural Problems.} Remove observations with missing firm identifiers or duplicate firm-year observations. This quality control check drops a negligible number of observations.

\paragraph{Step 2: Missing or Negative Values.} Drop observations with missing or negative values in key balance sheet and income statement variables. Since deflated monetary values should be non-negative, negative values indicate data errors. Checked variables include:
\begin{itemize}
\item Tangible fixed assets (physical capital)
\item Balance sheet intangibles (excluding goodwill)
\item Revenue, production value, gross value added, wagebill
\item Total assets, long-term debt, short-term debt, interest expenses
\item Tangible investment flows
\item Intangible construction inputs: R\&D expenditures, advertising, training
\end{itemize}

\paragraph{Step 3: Missing Economic Activity.} Drop observations indicating absence of genuine economic activity. Specifically, I remove firms with:
\begin{itemize}
\item Zero or missing number of workers
\item Less than 1,000 euros in tangible fixed assets
\item Less than 1,000 euros in revenue
\item Less than 1,000 euros in production value
\item Less than 500 euros in wagebill
\end{itemize}
These thresholds eliminate shell companies and data errors while retaining small active firms. Physical capital is defined as tangible fixed assets following standard practice in production function estimation, excluding inventories and working capital.

\paragraph{Step 4: Panel Structure.} Require at least two consecutive observations per firm to enable construction of intangible capital stocks via the perpetual inventory method (PIM). Firms with isolated observations cannot be used for dynamic capital stock accumulation.

\paragraph{Final Sample.} The sequential cleaning procedure yields a final analytical sample of \textbf{1,759,093 firm-year observations} spanning 2011-2022. All observations have complete data for core balance sheet variables, positive economic activity, and sufficient panel length for capital stock construction. Missing values in R\&D, advertising, or training expenditures are treated as zero investment in these categories, following standard practice in the intangible capital literature. 

\subsection{Intangible Capital Construction}

Following \citet{peterstaylor2017}, I construct intangible capital stocks using the perpetual inventory method (PIM). The baseline measure combines two components: knowledge capital from R\&D and balance sheet intangibles.

\paragraph{Knowledge Capital (from R\&D).} Accumulated from reported R\&D expenditures using sector-specific depreciation rates from \citet{ewenspeterswang2025}:
\begin{equation}
K^{knowledge}_{j,t} = (1-\delta^{sector}_{R}) K^{knowledge}_{j,t-1} + RD_{j,t},
\end{equation}
where $\delta^{sector}_{R}$ varies by Fama-French 5 industry classification:
\begin{itemize}
\item Consumer: $\delta_R = 0.43$
\item Manufacturing: $\delta_R = 0.50$
\item High Tech: $\delta_R = 0.42$
\item Health: $\delta_R = 0.33$
\item Other: $\delta_R = 0.35$
\end{itemize}
Initial stocks are set to zero ($K^{knowledge}_{j,2011} = 0$). The sector-specific rates capture heterogeneity in knowledge obsolescence across industries: manufacturing R\&D depreciates faster (50\%) than health-related R\&D (33\%), reflecting differences in product cycles and technological change.

\paragraph{Balance Sheet Intangibles.} Externally acquired intangibles recorded on firm balance sheets, measured directly as stock variables (excluding goodwill). These capture purchased patents, software, databases, and other codified intangible assets.

\paragraph{Total Intangible Capital.} The baseline measure sums knowledge capital and balance sheet intangibles:
\begin{equation}
K^{intangible}_{j,t} = K^{knowledge}_{j,t} + K^{external}_{j,t}.
\end{equation}
This \emph{BS+R\&D} measure focuses on intangibles with clearer market values and avoids the measurement challenges associated with organization capital (SG\&A-based accumulation). Total capital is $K^{total}_{j,t} = K^{physical}_{j,t} + K^{intangible}_{j,t}$, where physical capital equals tangible fixed assets from balance sheets.

\subsection{Winsorization}

To limit the influence of extreme outliers while preserving genuine economic variation, I apply winsorization to key variables by year and sector (3-digit CAE code). Financial variables and investment rates are winsorized at the 5th and 95th percentiles within each year-sector cell, allowing for industry heterogeneity in distributions while mitigating the impact of extreme values. This approach balances outlier treatment with preserving cross-sectional variation that is economically meaningful.

\subsection{Final Dataset Structure}

The final analysis-ready dataset contains 1,759,093 firm-year observations with:
\begin{itemize}
\item All monetary variables in constant 2020 prices
\item Constructed intangible capital stocks using sector-specific depreciation
\item Firm-level skill composition (share of workers with tertiary education, share of R\&D workers)
\item Multiple intangible capital definitions (BS+R\&D baseline, full Peters-Taylor alternative)
\item Investment rates and intensity measures (intangible intensity, R\&D intensity)
\item Financial constraint proxies (leverage, interest rates, credit spreads)
\item Winsorized versions of key variables
\end{itemize}


\section{Robustness: Alternative Outcome Measures}
\label{app:robustness}

This appendix presents robustness checks for the main empirical results using alternative outcome measures. The main text focuses on gross value added (GVA) as the preferred output measure because it captures the firm's contribution to aggregate output net of intermediate inputs. Here I show that the key findings are robust to using revenue or production value as alternative measures.

\subsection{Fact 1: Complementarity}

Table \ref{tab:complementarityrobust} replicates the complementarity analysis from Table \ref{tab:complementarity} using log revenue (columns 1--3) and log production value (columns 4--6) as dependent variables. Each panel follows the same nested specification: no controls, fixed effects only, and fixed effects with controls.

\begin{table}[htbp]\centering\small
\def\sym#1{\ifmmode^{#1}\else\(^{#1}\)\fi}
\caption{Complementarity Between Intangibles and Skilled Labor: Alternative Outcomes\label{tab:complementarityrobust}}
\begin{tabular}{l*{6}{c}}
\toprule
            &\multicolumn{3}{c}{Log Revenue}                                  &\multicolumn{3}{c}{Log Production}                               \\\cmidrule(lr){2-4}\cmidrule(lr){5-7}
            &\multicolumn{1}{c}{(1)}&\multicolumn{1}{c}{(2)}&\multicolumn{1}{c}{(3)}&\multicolumn{1}{c}{(4)}&\multicolumn{1}{c}{(5)}&\multicolumn{1}{c}{(6)}\\
\midrule
Intangible Intensity&       0.64\sym{***}&      -0.05\sym{***}&      -0.10\sym{***}&       0.48\sym{***}&      -0.05\sym{***}&      -0.11\sym{***}\\
            &     (0.01)         &     (0.01)         &     (0.01)         &     (0.01)         &     (0.01)         &     (0.01)         \\
\addlinespace
Share Skilled Workers&       0.18\sym{***}&      -0.05\sym{***}&      -0.01\sym{***}&       0.36\sym{***}&      -0.05\sym{***}&      -0.01\sym{**} \\
            &     (0.00)         &     (0.00)         &     (0.00)         &     (0.00)         &     (0.00)         &     (0.00)         \\
\addlinespace
Intangible Intensity $\times$ Share Skilled&       0.98\sym{***}&       0.23\sym{***}&       0.03\sym{*}  &       1.18\sym{***}&       0.25\sym{***}&       0.04\sym{***}\\
            &     (0.03)         &     (0.02)         &     (0.01)         &     (0.03)         &     (0.02)         &     (0.01)         \\
\midrule
Observations&   1,759,093         &   1,759,093         &   1,759,084         &   1,759,093         &   1,759,093         &   1,759,084         \\
Adjusted R-squared&       0.012         &       0.933         &       0.952         &       0.017         &       0.931         &       0.951         \\
Firm FE     &          No         &         Yes         &         Yes         &          No         &         Yes         &         Yes         \\
Year FE     &          No         &         Yes         &         Yes         &          No         &         Yes         &         Yes         \\
Industry FE &          No         &         Yes         &         Yes         &          No         &         Yes         &         Yes         \\
Controls    &          No         &          No         &         Yes         &          No         &          No         &         Yes         \\
\bottomrule
\multicolumn{7}{l}{\footnotesize Intangible intensity = $K_{intangible}$ / $K_{total}$.}\\
\multicolumn{7}{l}{\footnotesize Controls include log total capital, log employment, and log firm age.}\\
\multicolumn{7}{l}{\footnotesize Robust standard errors in parentheses.}\\
\end{tabular}
\end{table}


The interaction between intangible intensity and skilled labor share is positive and statistically significant across all specifications and outcome measures. The magnitude of the complementarity effect is similar across GVA, revenue, and production, confirming that the technological complementarity documented in the main text is not driven by the choice of output measure.

\subsection{Fact 3: Underexploitation by Leverage}

Table \ref{tab:underexploitationrobust} replicates the underexploitation analysis from Table \ref{tab:underexploitation} using log revenue (columns 1--3) and log production value (columns 4--6) as dependent variables. Each panel reports results for all firms, low-leverage firms, and high-leverage firms.

\begin{table}[htbp]\centering\small
\def\sym#1{\ifmmode^{#1}\else\(^{#1}\)\fi}
\caption{Underexploitation of Complementarity: Alternative Outcomes\label{tab:underexploitationrobust}}
\begin{tabular}{l*{6}{c}}
\toprule
            &\multicolumn{3}{c}{Log Revenue}                                  &\multicolumn{3}{c}{Log Production}                               \\\cmidrule(lr){2-4}\cmidrule(lr){5-7}
            &\multicolumn{1}{c}{All}&\multicolumn{1}{c}{Low Lev.}&\multicolumn{1}{c}{High Lev.}&\multicolumn{1}{c}{All}&\multicolumn{1}{c}{Low Lev.}&\multicolumn{1}{c}{High Lev.}\\
\midrule
\textit{Main effects:}&                     &                     &                     &                     &                     &                     \\
\addlinespace
Intangible Intensity&      -0.10\sym{***}&      -0.11\sym{***}&      -0.10\sym{***}&      -0.11\sym{***}&      -0.12\sym{***}&      -0.11\sym{***}\\
            &     (0.01)         &     (0.01)         &     (0.01)         &     (0.01)         &     (0.01)         &     (0.01)         \\
\addlinespace
Share Skilled Workers&      -0.01\sym{***}&      -0.02\sym{***}&      -0.01         &      -0.01\sym{**} &      -0.02\sym{***}&      -0.00         \\
            &     (0.00)         &     (0.01)         &     (0.01)         &     (0.00)         &     (0.01)         &     (0.00)         \\
\addlinespace
\textit{Complementarity:}&                     &                     &                     &                     &                     &                     \\
\addlinespace
Intangible Intensity $\times$ Share Skilled&       0.03\sym{*}  &       0.04\sym{**} &      -0.00         &       0.04\sym{***}&       0.07\sym{***}&       0.00         \\
            &     (0.01)         &     (0.02)         &     (0.02)         &     (0.01)         &     (0.02)         &     (0.02)         \\
\midrule
Observations&   1,759,084         &     852,528         &     856,243         &   1,759,084         &     852,528         &     856,243         \\
Adjusted R-squared&       0.952         &       0.957         &       0.958         &       0.951         &       0.955         &       0.959         \\
\bottomrule
\multicolumn{7}{l}{\footnotesize Low- and high-leverage defined relative to sector-year median leverage.}\\
\multicolumn{7}{l}{\footnotesize All specifications include firm, year, and industry fixed effects.}\\
\multicolumn{7}{l}{\footnotesize Controls include log total capital, log employment, and log firm age.}\\
\multicolumn{7}{l}{\footnotesize Robust standard errors in parentheses.}\\
\end{tabular}
\end{table}


The pattern of underexploitation is robust across outcome measures: the complementarity coefficient is positive and significant for low-leverage firms but small and insignificant for high-leverage firms. This confirms that the differential ability to exploit intangible-skill complementarity by financial constraint status is not an artifact of the GVA measure.


\section{Computational Methods}
\label{app:computation}

This appendix describes the numerical methods used to solve and simulate the quantitative model. The computational implementation follows standard practices in the heterogeneous firm literature while incorporating several optimizations to handle the four-dimensional state space efficiently.

\subsection{Solution Algorithm Overview}

The model is solved by finding a stationary recursive competitive equilibrium. The algorithm consists of three nested loops:
\begin{enumerate}[label=(\roman*)]
\item \textbf{Outer loop}: Iterate on wages $(w_L, w_H)$ until labor markets clear
\item \textbf{Middle loop}: For given wages, compute the stationary distribution of firms over the state space
\item \textbf{Inner loop}: Solve the firm's dynamic optimization problem via value function iteration
\end{enumerate}

The outer loop updates wages based on excess labor demand until equilibrium is achieved within tolerance. For each wage guess, the inner loops solve the firm problem and compute the implied stationary distribution and aggregate labor demands.

\subsection{State and Choice Spaces}

\paragraph{State Variables.} The firm's state is $(z, K, S, D_{-1})$ where $z$ is idiosyncratic productivity, $K$ is tangible capital, $S$ is intangible capital, and $D_{-1}$ is debt inherited from the previous period. Idiosyncratic productivity follows an AR(1) process in logs and is discretized using the \citet{tauchen1986} method with evenly-spaced grid points in log space covering multiple unconditional standard deviations of the process. The continuous state variables are discretized on grids:
\begin{itemize}
\item Tangible capital $K$: Exponentially spaced grid providing finer resolution at low capital levels where marginal products are highest
\item Intangible capital $S$: Exponentially spaced grid (similar to tangible capital) to capture high curvature in marginal products when $S$ is low, particularly important given the complementarity structure
\item Debt $D$: Linearly spaced grid from zero to an upper bound
\end{itemize}

\paragraph{Choice Variables.} The firm chooses tangible investment $I^K$, R\&D labor $H^R$, and new debt $D'$. Static labor inputs $(L, H^P)$ are determined by intratemporal first-order conditions given the state and wages, solved via nested bisection. The investment choices are discretized on grids:
\begin{itemize}
\item Tangible investment $I^K$: Linearly spaced grid allowing both positive investment and bounded disinvestment
\item R\&D labor $H^R$: Two-segment exponentially spaced grid with finer resolution at low values to capture high curvature in the R\&D production function when returns to scale are less than unity
\end{itemize}

New debt $D'$ is computed analytically rather than via grid search. Given investment choices $(I^K, H^R)$, the collateral constraint and dividend non-negativity jointly determine optimal borrowing as the minimum of the financing gap and available collateral.

\subsection{Value Function Iteration with Policy Improvement}

The Bellman equation is solved via value function iteration enhanced with Howard's policy improvement algorithm. The algorithm alternates between two types of steps:

\paragraph{Policy Improvement.} Every several iterations, the algorithm performs full optimization over the choice space to update policy functions. This step searches over the discrete choice grids for $(I^K, H^R)$ while computing $D'$ analytically subject to the collateral constraint $D' \leq \alpha_K K' + \alpha_S S'$ and dividend non-negativity $\text{Div} \geq 0$. This step is computationally intensive as it requires evaluating all choice combinations at each state point.

To escape potential local optima, the algorithm periodically reverts to searching the entire choice grid rather than using local search. Between these full searches, subsequent policy improvements employ local search around the previous optimal policy, examining only a small neighborhood of grid points. This substantially reduces the effective choice space while maintaining solution accuracy.

\paragraph{Policy Evaluation.} Between policy improvements, the value function is updated using the \emph{fixed} policy functions without re-optimization. This step evaluates the value function given the stored optimal policies, requiring only continuation value computation rather than choice optimization. Multiple policy evaluation steps can be performed between each policy improvement, dramatically reducing computational cost while maintaining convergence.

The value function is initialized with approximate stationary values based on perpetuity calculations rather than zero, which avoids a ``cold start'' problem where zero continuation values make future investments appear worthless.

\subsection{Computational Optimizations}

Several optimizations reduce computation time substantially:

\paragraph{Parallelization.} The state space loops are parallelized using OpenMP directives. Since the Bellman equation evaluation at each state point is independent conditional on the value function, outer loops can be executed in parallel across CPU cores.

\paragraph{Precomputed Static Labor.} The static labor choices $(L^*, H^{P*})$ depend only on $(z, K, S)$ and wages, not on debt $D_{-1}$. These solutions and associated output and gross profits are precomputed for all productivity-capital triplets before each policy improvement step, eliminating redundant computation across the debt dimension.

\paragraph{Feasibility Screening.} Investment choices that would violate capital grid bounds or the dividend constraint are identified and skipped immediately, avoiding unnecessary continuation value evaluations.

\paragraph{Robust Static Labor Solution.} Static labor demands are solved via nested bisection rather than gradient-based methods. For each candidate level of unskilled labor $L$, optimal skilled production labor $H^P$ is found by bisecting on the skilled labor first-order condition. Then unskilled labor $L$ is determined by bisecting on the unskilled labor first-order condition evaluated at the optimal $H^P(L)$. This approach ensures robustness for firms with extreme state combinations.

\subsection{Interpolation}

Continuation values require evaluating the value function at next-period state points $(K', S', D')$ that typically do not lie on grid nodes. Trilinear interpolation is employed across the three continuous state dimensions, with grid location performed via binary search for computational efficiency.

\subsection{Stationary Distribution}

The stationary distribution $\Psi(z, K, S, D)$ is computed via forward iteration accounting for firm entry and exit. Each iteration distributes firm mass according to:
\begin{enumerate}[label=(\roman*)]
\item Survivors transition according to policy functions and productivity shocks, with survival probability $1-\zeta$
\item Exiters are replaced by entrants starting at the minimum capital values with zero debt, with productivity drawn from the stationary distribution of the AR(1) process
\end{enumerate}

Since policy functions map states to off-grid points, firm mass is distributed to neighboring grid nodes using trilinear interpolation weights. The algorithm includes safeguards to prevent mass leakage when policies approach or exceed grid boundaries.

An optimized sparse representation tracks only states with non-trivial mass above a threshold, substantially accelerating distribution iteration when the equilibrium distribution is concentrated on a subset of the state space.

\subsection{Equilibrium Computation}

The outer wage iteration employs a dampened fixed-point updating rule. Given aggregate labor demands $L^d$ and $H^d$ computed by integrating firm-level demands over the stationary distribution, wages are updated proportionally to excess demand. The dampening parameter balances convergence speed against stability. Convergence is declared when both labor markets clear within tolerance.

\subsection{Implementation}

The model is implemented in Fortran 90 with double precision arithmetic throughout. The code employs the Intel Fortran compiler with full optimization and OpenMP enabled.

\bibliographystyle{apalike}
\bibliography{references}

\end{document}